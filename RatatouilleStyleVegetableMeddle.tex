\documentclass [11pt, letterpaper] {article}
\input {headings}
\newcommand \recipeName {Ratatouille-Style Vegetable Meddle}
\newcommand \fileName {RatatouilleStyleVetableMeddle}
\chead {\recipeName}

\begin {document}
\input {title}

I have a recipe for \href{GrilledRatatouille.html}{Grilled Ratatouille} that is great to prepare later in the Summer when zucchinis are plentiful. This is a variation using mushrooms and a technique that I developed for \href{RustRoast.html}{Rust Roast} when making the tomato sauce. This recipe can be prepared completely indoors by frying the eggplants in olive oil, peeling the bell peppers with a small sharp knife and frying them in the same pan, and then frying the onions, then the mushrooms, and making the tomato sauce, all in the same pan. Each vegetable is cooked separated and put on a basin so that the meddle is only mixed in the end. In this version, which I wrote for a traditional student party ``Cooking in Nelson's Kitchen'' mixes cooking on the grill and on a pan to make it possible to cook outdoors during the times when we are still concerned with COVID transmission indoors. 
\vspace{0.5in}

\begin{description}

\item[Ingredients:]\ \\
	\begin{itemize}
	\item Extra-virgin olive oil
	\item 3 large yellow onions 
	\item 2 large cans of tomatoes (good quality Italian canned tomatoes) 
	\item 1 small jar of roasted and peeled red bell peppers (or three fresh red bell peppers that you roast and peel yourself) 
	\item 2 large eggplants, or 3 small ones
	\item 1 pound of cremini or white mushrooms
	\item 2 tablespoons of soy sauce
	\item 1 tablespoon of dried tarragon (or dried basil)
	\item 1 hot pepper (if in Brazil, use two pimenta de cheiro)
	\item 1 large fresh jalape\~no pepper (optional)
	\item Freshly ground black pepper
	\item Fresh Basil (or fresh Italian Parsley)
	\end{itemize}

\item[Procedure:]\ \\
	\begin{enumerate}
	\item {\bf Prep the eggplant}
	\begin{itemize}
	\item Cut the ends of the eggplants.
	\item Peel strips of the skin of the eggplant, leaving some skin on them.
	\item Slice eggplants lengthwise in 1 cm thick slices.
	\item Sprinkle eggplant slices with salt and let them sit for at least half hour.
	\end{itemize}
	\item {\bf Prep bell peppers}
	\begin{itemize}
	\item Cut both ends of the bell peppers.
	\item Placing each bell pepper standing on an end, cut into three or four flat pieces so that the skin is on the outside of each piece and you can easily remove the seeds.
	\item Lightly sprinkle the inner side of the pepper of each slice with salt.
	\end{itemize}
	\item {\bf Grill and Peel bell peppers}
	\begin{itemize}
	\item On a hot gas grill, place the bell peppers skin side down and grill until the skin is completely black and large bubbles form between the skin and the pepper. Do not overcook so that the peppers are not dried out.
	\item Transfer the peppers from the grill to a container with a lead, cover and let it cool covered until they are warm to the touch.
	\item Peel the skins off the peppers.
	\item Dice peppers in 1/2 in (1.5 cm) dices and add to the container that will contain the meddle in the end.
	\end{itemize}
	\item {\bf Prep the Mushrooms}
	\begin{itemize}
        		\item Wash the mushrooms.
		\item Cut mushrooms into 1/2 in (1.5 cm) dice.
		\item Toss mushrooms with soy sauce.
	\end{itemize}
	\item {\bf Prep the Onions}
	\begin{itemize}
		\item Peel the onions.
		\item Dice into 1 cm dice.
	\end{itemize}
	\item {\bf Prep the Tomatoes}
		\begin{itemize}
		\item Place a large strainer on top of a large bowl.
		\item Open the cans of whole tomatoes and empty them over the strainer.
		\item Using a pair of kitchen scissors, roughly cut up the tomatoes to release their juices.
		\item Toss the cut up tomatoes in the strainer with a rubber spatula.
		\item Let sit for at least half hour for tall the excess juice to drain from the tomatoes.
	\end{itemize}
	\item {\bf Sautee the Onions}
	        \begin{itemize}
	        \item Place a large heavy bottom stainless steel pot over high heat.
	        \item Add about 1/2 cup of olive oil to the pot.
	        \item Add all the onions at once.
	        \item Sautee the onions until they are translucent, but not browning.
	        \item Using a spider strainer, lift the onions from the oil and put them in the meddle bowl.
	        \end{itemize}
	\item {\bf Sautee the Mushrooms}
		\begin{itemize}
		\item Add all the mushrooms at once to the oil where the onions had cooked.
		\item Cover the pan and let them cook for a while. They will release water. Stir from time to time.
		\item Once oil is visible in the pan again and the mushrooms start browning, add the dried tarragon. Crush the tarragon (or basil) on the palms of your hand over the mushrooms.
		\item Stir for a few more minutes.
		\item Using a spider strainer, lift the mushrooms from the oil and put them in the meddle bowl.
		\end{itemize}
	\item {\bf Prepare the tomato sauce}
		\begin{itemize}
		\item There should still be enough oil in the pan to coat the bottom of the pan. If there is not enough, add a small amount of olive oil.
		\item Once the oil is hot, add the drained tomatoes to the pan. 
		\item Using a flat wooden spoon, keep stirring the tomatoes over high heat until they start sticking to the bottom of the pan and you can smell a distinct smell of roasted tomatoes.
		\item This process may take some time. It is important to cook the tomato solids until they have dried out and are separating from the oil. However, you need to be careful to not burn the tomatoes.
		\item Add all the liquid that had drained to the tomatoes to the pan.
		\item If using hot peppers (pimenta de cheiro is preferred) add to the pan  --- alternatively you can add fresh diced jalape\~no peppers to the meddle at the end.
		\item Let the tomato sauce cook until it has thickened some.
		\item Transfer the tomato sauce to the meddle bowl.	
		\end{itemize}
	\item {\bf Grill the Eggplants}
	\begin{itemize}	
		\item Get your grill as hot as you can (my gas grill gets to 550F)
	 	\item Brush the grill with a hard grill brush to make sure it is clean
	 	\item Pour a small amount of oil in a small dish, fold a piece of paper towel several times and holding the folded paper towel with kitchen tongues, deep it in the oil and smear all over the grill. Cover the grill to let the oil burn for a minute. Repeat the process three or four times to reduce the stickiness of the grill. 
		\item Reduce the temperature of the grill to a moderate high heat.
		\item Dry the slices of eggplant  with paper towels.
		\item Pour some olive oil into a dinner plate.
		\item Quickly coat both sides of each eggplant slice on the olive oil in the plate. Keep adding more oil to the plate as you need. You may want to coat as many slices as will fit in your grill before you start grilling.
		\item Grill the eggplants until they are soft, flipping them with a metal spatula as needed. 
		\item When removing from the grill put in a covered bowl so that they continue cooking as they cool down.
		\item When they are cool enough to handle, cut the eggplant slices into a large dice.
		\item Add the diced grilled vegetables, along with any juices that they have released, to the meddle bowl.
	\end{itemize}
	\item{\bf Basil (or parsley) and Serving}
	\begin{itemize}	
		\item Chop the basil, or parsley, and mix into the meddle.
		\item If using, add finely diced fresh jalape\~no to the meddle.
		\item Add freshly ground pepper (to taste).
		\item Gently stir the meddle and serve.
	\end{itemize}	
     	\end{enumerate}         
\end{description}
\end{document}



