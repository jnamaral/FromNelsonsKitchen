\documentclass[11pt,letterpaper]{article}
\input{headings}
\newcommand \recipeName {Harissa Tomato Sauce}
\chead{\recipeName}

\begin{document}
\input{title}

This is the tomato sauce for the \href{SpicedLentilsWithPumpkin.html}{Spiced Lentils with Pumpkin}, a Moroccan recipe by Tess Mallos in {\it The food of Morocco: a journey for food lovers}. This is a simple but very tasty tomato sauce. It can be used in many other ways too.  

\begin{description}

\item[Ingredients:]\ \\
	\begin{itemize}
	\item 1 teaspoon ground turmeric 
	\item 2 teaspoons of paprika
	\item 1 teaspoon of ground cumin
	\item 6 tablespoons of olive oil
	\item 2 onions, finely chopped
	\item 6 garlic cloves, finely chopped
	\item 2 teaspoons of \href{Harissa.html}{harissa}
	\item 2 tablespoons of tomato paste
	\item 1 large can (16 oz) of tomatoes 
	\item 1 1/2 teaspoon of salt
	\item 2 tablespoons of chopped flat-leaf (Italian) parsley
	\item 4 tablespoons of chopped cilantro
	\end{itemize}

\item[Procedure:]\ \\
	\begin{itemize}
	\item Measure ground spices in a small bowl.
	\item Heat the oil in a large saucepan over low heat.
	\item Add the chopped onions and cook until softened.
	\item Add the chopped garlic and cook for a few seconds.
	\item Add the ground spices and the harissa.
	\item Cook stirring for about 30 seconds.
	\item Add the tomato paste and cook stirring for about 30 seconds.
	\item Add the tomatoes and salt, half the parsley and half of the cilantro.
	\item Cover, cook in moderate heat for about 20 minutes until it acquires the consistency of a course tomato sauce.
	\item Remove from heat and let it cool until it is warm.
	\item	Add the remaining chopped parsley and cilantro.
	\item If not using immediately, put in a covered glass container and refrigerate.
	\item It will keep well in the refrigerator for two or three days.
	\item Reheat gently before using.
	\end{itemize}
\end{description}
\end{document}
