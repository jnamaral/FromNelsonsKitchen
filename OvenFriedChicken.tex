\documentclass[11pt,letterpaper]{article}
\input{headings}
\newcommand \recipeName {Oven-Fried Chicken}
\newcommand \fileName {OvenFriedChicken}
\chead{\recipeName}

\begin{document}
\input{title}

Often I buy large packages of boneless skinless chicken breasts to grill or to cook in a pan. I brine and freeze the individual chicken breasts to have them ready for use later. However the chicken breast are not in a good shape for cooking. Thus, I remove the tenderloins, I cut off the triangular thin end of the breasts. If the breasts are specially thick, I put them in their sides and cut a slice from the under side of the thick end. The goal is to make the chicken breast more even for cooking in the grill or in a pan. For example, see the recipe for \href{SlicedGrilledChicken.html}{Sliced Grilled Chicken} for an use of the breasts. This recipe is what I do with the tenderloins and the bits of chicken breasts that I left from the trimming. You can also prepare it with whole breasts, ideally cut them into strips to increase the amount of crisp bits. I have also done this very well with boneless skinless chicken thighs.
\begin{description}

\item[Ingredients:]\ \\
	\begin{itemize}
	\item 3 to 4 pounds of boneless skinless chicken breasts
	\item salt
	\item sugar
	\item white sandwich bread
	\item 3/4 cup of all-purpose flour
	\item 1/2 teaspoon of white pepper
	\item 1 teaspoon of paprika
	\item 1 teaspoon of garlic powder
	\item 1/2 cup of extra-virgin olive oil
	\item 1/4 cup of freshly grated parmesan cheese
	\item freshly Ground Pepper
	\item 1 tablepoon of dried time
	\item 2 tablespoons of mayonnaise
	\item 1 tablespoon of Dijon mustard
	\item 2 tablespoon of water
	\item 4 eggs 
	\item cooking spray
	\end{itemize}

\item[Important Equipment:]\ \\
	\begin{enumerate}
	\item Food processor
	\item Large rimmed backing sheet
	\item Rack to fit on backing sheet
	\end{enumerate}
\item[Procedure:]\ \\
	\begin{enumerate}
	\item {\bf Trim the chicken breasts}
	\begin{itemize}
	\item Remove the tenderloins.
	\item Cut a small triangular shape from the end of each breast to remove the thin end of the breast.
	\item Put each breast on its side and cut a thin slice from the thickest part to make each breast a bit more even.
	\item If you will use the whole breasts for this recipe, cut them crosswise in one-inch wide strips.
	\item Otherwise save the breasts for another use and use the trims for this recipe.
	\item Wash the chicken breast throughly in cold running water.
	\end{itemize}
	\item {\bf Brining the chicken}
	\begin{itemize}
	\item Make a brining solution with the following proportions: for each  two quarters of water, add 1/4 cup of table salt and 1/4 cup of sugar.
	 \item Add all the chicken to the brine and let sit, in the refrigerator, for at least two hours, but you can also keep them in the brine overnight.
	 \item Remove the chicken from the brine and put in a colander or over a rack inside a clean sink to let all the brine run out. 
	 \item If you are doing the brining a day or two ahead, put the drained chicken into the fridge and leave it uncover for several hours to help it keep drying. After 10 or 12 hours, cover the chicken so that it does not get too dry in the fridge.
	 \end{itemize}
	  \item {\bf Prepare the bread crumbs}
	  \begin{itemize}
	  \item Turn the oven on to 325 F.
	  \item Process the sandwich bread in the food processor to obtain coarse crumbs. Depending on the size of your food processor, you may need to do the processing in two or three batches.
	  \item Spread the crumbs in two rimmed baking sheets.
	  \item Toast the bread crumbs, make sure to rotate the pans after 15 minutes, and every 10 minutes afterwards. Also, using a spoon, stir the bread crumbs in the pan from time to time for an even toasting.
	  \item Breadcrumbs are done when they are a light golden colour.
	  \item Remove from the oven and let it cool completely in the baking sheets. This can be done a day ahead.
	 \end{itemize}
	\item {\bf Season the flour}
	\begin{itemize}
	\item In a wide dish that will make it easy for the breading, mix the flour, white pepper, paprika, and garlic powder.
	\end{itemize}
	\item {\bf Season the bread crumbs}
	\begin{itemize}
	\item Put the bread crumbs in a large bowl.
	\item Slowly drizzle some of the olive oil on top to create a few circles of olive oil. Stir well, but gently, with a large spoon. Repeat the process until you have used about 1/2 cup of olive oil.	
	\item Freshly grate the parmesan cheese over the breadcrumbs and add a few grindings of black pepper.
	\item Put the dried time on the palm of your hand and rub it with your other palm on top of the breadcrumb mixture.
	\item Stir well with a spoon.
	\end{itemize}
	\item {\bf Prepare the egg mixture}
	\begin{itemize}
	\item In a shallow dish that will make it easier to do the breading, add the mayonnaise, the mustard, and the two tablespoons of water.
	\item Using a whisk stir it well until it forms an homogeneous mixture.
	\item Add the eggs and stir well until well incorporated.
	\end{itemize}
	\item {\bf Bread the chicken}	
	\begin{itemize}
	\item If the chicken is still quite wet from the brining, use some paper towels to remove the excess moisture.
	\item Over the sink, throughly spray the rack that you will fit over the rimmed baking sheet.
	\item If you wish to make cleanup easier, you may lineup the baking sheet with foil before fitting the rack on it.
	\item Coat each piece of chicken with the seasoned flour, then with the egg mixture and then with the seasoned bread crumbs.
	\item Make sure to path the bread crumbs into the chicken with your hands after the initial coating to ensure that every bit of every piece is well coated.
	\item After breaded, the chicken can be kept in the fridge, uncovered, for up to a day.
	\end{itemize}
	\item {\bf Oven-fry the chicken}
	\begin{itemize}
	\item Pre-heat the oven to 400 F.
	\item Put the chicken in the oven.
	\item Rotate trays after 15 minutes. If cooking two trays, you need to switch their position in the oven, and also rotate them front-back.
	\item The chicken should be done in about 25 minutes. You can remove a thick piece from the oven and slice through the middle to check. It should be cooked through but still very moist inside.
	\item Let it cool for five minutes and serve.
	\end{itemize}	
      	\end{enumerate}         
\end{description}
\end{document}



