\documentclass[11pt,letterpaper]{article}
\input{headings}
\newcommand \recipeName {Vatap\'a}
\chead{\recipeName}

\begin{document}
\input{title}

Vatap\'a is a dish from Bahia in the Northeast of Brazil. The roots are African. The base for the dish is bread soaked in milk, and then it has a very interesting combination of ingredients typical of that region of Brazil. Vatap\'a is a traditional filling for Acaraj\'e. You can usually find frozen raw grated cassava at SuperStore.

\vspace{0.3in}

\begin{description}

\item[Ingredients:]\ \\
	\begin{itemize}
	\item	milk
	\item 1 french baguette or Italian bread
	\item small amount of dried shrimp
	\item 3 tbsp of unsweetened peanut butter  
	\item 250 grams of raw grated cassava
	\item 1 large onion
	\item 3 garlic cloves
	\item 2 inches of ginger root
	\item 8 whole tomatoes from a can
	\item 1 can of coconut milk
	\item fresh parsley
	\item 1/2 cup of dend\^e oil
	\item 1 pound of fresh shrimp
	\end{itemize}

\item[Procedure:]\ \\

	\begin{enumerate}
	\item {\bf Preparing the base}
	\begin{itemize}
	\item Cut the bread in slices, put in a large bowl and pour enough milk to soak the bread completely
	\item Peel and cut the onion in thick slides.
	\item Peel and cut the ginger into slices
	\item Peel the garlic cloves
	\item Add the onions, ginger, and garlic to the bowl of a large food processor and process it until liquified. 
	\item Add the soaked bread, the grated cassava, the peanut butter, and the tomatoes to the food processor.
	\item Process, scraping the sides of the bowl from time to time, until you have an homogeneous paste.
	\end{itemize}
	
	\item {\bf Cook the base}
	\begin{itemize}
	\item Transfer the paste to a heavy-bottom pan and cook at moderate heat, stirring the pot from time to time to avoid sticking to the pan and burning, until the paste becomes thick and start to release from the bottom of the pan.
	\end{itemize}
	
	\item {\bf Finish the vatap\'a}
	\begin{itemize}
	\item Stir in the dried shrimp, coconut milk, the dend\^e oil and let the vatapa get very hot again, but without boiling
	\item Turn off the heat, add the fresh shrimp and the chopped parsley and cover the pot for 3-5 minutes until the shrimp is cooked in the residual heat.
	\end{itemize}

	\end{enumerate}
\end{description}
\end{document}



