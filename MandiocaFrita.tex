\documentclass[11pt,letterpaper]{article}
\input{headings}
\newcommand \recipeName {Mandioca Frita}
\chead{\recipeName}

\begin{document}
\input{title}

{\em Mandioca Frita} is a traditional appetizer served throughout the
South of Brazil. The freshly cooked yucca with the steam raising from
the dish is a very popular side dish to serve with grilled meats. The
fried yucca is usually prepared the day after the yucca was cooked and
served plain boiled. The yucca that we find in supermarkets in North
America is waxed for preservation. When buying try to select roots
that are not bruised and that do not have dark spots. Always buy extra
because you will waste some. Prefer the middle thickness as often the
very thick specimens are woody.

When harvested fresh from the soil, the yucca is very easy to peel. It
has a double skin, a very thin paper like skin and a thicker white
skin underneath. You have to remove both of them. In the freshly
harvested yucca the white skin separates from the center of the root
effortlessly. I used to have a very difficult time peeling the waxed
yucca that we find in North America because the skin seems to be
firmly attached to the center. Now I discovered that soaking the roots
for a few minutes in very warm water expands the skin and loosens it
from the root, making it almost as easy to peel as the freshly
harvested roots.

A much easier alternative is to buy frozen yucca that has already been peeled. You can find it at Para\'iso Tropical, SuperStore and sometimes at Chinese stores such as Lucky 97.

\begin{description}

\item[Ingredients:]\ \\
	\begin{itemize}
	\item Yucca Root
	\item Canola Oil
	\item Salt
	\item Water
	\item Mayonnaise
	\item Ketchup
	\item Mustard
	\end{itemize}

\item[Procedure:]\ \\
	\begin{enumerate}
	\item {\bf Peel the yucca}
	\begin{itemize}
	\item Fill your sink with hot tap water and soak the yucca root in warm water for a few minutes to loosen the skin.
	\item Trim the ends of each yucca root with a chefs knife.
        \item With a pointed paring knife peel the yucca starting at the top. 
	\item Cut the yucca into 3 to 4 inch long segments.
	\item Stand each segment up and cut in half lengthwise.
	\item Cut each half in half again.
	\item Remove the center string from each segment.
	\item Remove any brown, black, stringy or woody parts.
	\item At this point you may submerge the peeled yucca in fresh water so that all the pieces of yucca are covered by water, and put in the freezer, or you may proceed to cook it.
	\end{itemize}
	\item {\bf Cook the yucca}
	\begin{itemize}
	\item If you froze the yucca, drop the frozen yucca with the surrounding water in a pressure cooker, adding enough water for the yucca to remain submerged in water after it taws.  
	\item Otherwise drop the peeled yucca in a pressure cooker and cover with water. Do not add any seasoning or salt to the water.
	\item Cover the pressure cooker, bring to a boil and cook under pressure for 15 minutes.
	\item Put the pressure cooker under running cold water and carefully move the pressure valve to release the steam.
	\item Turn off the fire, add salt to the water and let it stand for 5 to 10 minutes.
	\item Drain the cooked yucca in a colander.
	\item Often the boiled yucca is served as a side dish to meat dishes as is. 
	\end{itemize}
\newpage
	\item {\bf Pre-frying the yucca}
	\begin{itemize}
	\item The best is to fry the yucca when it has dried well after it was boiled. Typically in Brazil you would fry the yucca the day after you boiled it.
	\item Heat up pure canola oil in a deep fryer. Add enough oil to submerge the pieces of yucca in the oil.
	\item Add a half dozen pieces of yucca at a time to the fryer and cook it until the pieces acquire a light tan color. You will finish frying later.
	\item Repeat the process until you have pre-fried all the pieces of yucca.
	\end{itemize}
	\item {\bf Preparing the sauces}
	\begin{itemize}
	\item In a small dish mix mayonnaise with ketchup to make a rose sauce.
	\item In a second small dish, mix mayonnaise with mustard.
	\end{itemize}
	\item {\bf Finish frying the yucca}
	\begin{itemize}
	\item When your guests are assembled, warm up the canola oil again.
	\item Return the pieces of yucca (6 or 8 at a time) to the hot oil and finish frying them until they acquire a beautiful tanned color. You have to watch closely at this point because they will cook quickly. 
	\item Remove the fried yucca to a paper towel or a cooling rack.
	\item Lightly sprinkle them with salt while they are still hot.
	\item Serve as an appetizer with the sauces.
	\end{itemize}
	\end{enumerate}
\end{description}
\end{document}




