\documentclass[11pt,letterpaper]{article}

\newcommand \recipeName {Daniel's Meat Sauce}
\input{headings}
\begin{document}
\input{title}

Daniel is milk intolerant and he liked bolognese sauce. This recipe started as a way to feed him some protein and it has become the most important staple in our house. I make very large batches and then freeze in small containers for individual serving portions. It can be microwaved from the freezer to be served with pasta, rice or with flour tortilla. A very successful recipe is ``tortilla pasta'' which I provide as a separate recipe. I have been making this meat sauce for many years and over time I have improved it to give it more meat flavour and also to make it easier to prepare. For a large batch, it takes a long time to make (approximately two days of slow oven cooking). But most of the time it is not monitored. Daniel does not like to see vegetables or onions in his sauce. Therefore I do process all the vegetables, including the can tomatoes in the food processor to make sure that the sauce has no chunks of anything. I have large ovens and large dutch oven pans. Before you make, check your equipment and oven to make sure that you can fit everything and adjust the recipe accordingly. I also have very large stainless steel bowls (restaurant size) which makes it much easier to mix the meat with the wine and seasonings before starting. The quantities are an approximation. You can vary the amount of meat, tomatoes, onions, etc, and still have a very nice product. I do not add garlic to this base sauce to make it more versatile. I may add garlic to pasta for example when serving with this sauce. There is no pepper either. Again add freshly ground black pepper to the dish in which you use the sauce. In Edmonton I always buy "La Provencela" can tomatoes in the Italian Centre.

\begin{description}

\item[Ingredients:]\ \\
	\begin{itemize}
	\item 4 pounds ground beef
	\item 4 pounds ground pork
	\item 1/4 cup of soy sauce
	\item 2 Tbs prepared Dijon mustard
	\item 1 bottle of dry white wine
	\item 6 dry bay leaves
	\item 8 16 oz can of whole tomatoes
	\item 3 pounds of onions
	\item 1 pound of carrots
	\item 1 pound of celery
	\item 3 Tbs of extra-virgin olive oil
	\end{itemize}
\item[Equipment and Supplies:]\ \\
	\begin{itemize}
	\item three large dutch oven that fit in your oven at the same time -- two can be replaced by oven-proof bakerware. It is important that all containers be non-reactive (cast iron, for example, cannot be used)
	\item a large food processor
	\item a large mixing bowl
	\item a rubber spatula
	\end{itemize}

\item[Procedure:]\ \\
	\begin{enumerate}
	\item {\bf Mix the Meat:}
		\begin{itemize}
		\item Put the two ground meat, the wine, the soy sauce, and the mustard in a large mixing bowl. Gently mix. The best is to do it with you hands while keeping your fingers spread widely open. You want to loosen the meat with the wine and {\bf not} to press the meat together or to knead it. You may prefer to use latex gloves for this task.
		\item Transfer the meat mixture to a dutch oven.
		\item Nestle the bay leaves in the middle of the meat.
		\item Put the dutch oven, uncovered, in the oven.
		\end{itemize}
	\item {\bf Prep Vegetables}
		\begin{itemize}
		\item Peel the carrots and cut them lengthwise. If the carrots are thick, cut in quarters.
		\item Peel the onions and cut lengthwise in large pieces.
		\item Wash the celery and cut in large pieces.
		\item Put the vegetables in either a dutch oven or a bakerware, sprinkle lightly with salt.
		\item Add the olive oil and mix well.
		\item Put, uncovered, in the oven.
		\end{itemize}
	\item{\bf Tomatoes}
		\begin{itemize}
		\item Open the cans of tomatoes and put in the third container. If you cannot fit all the cans in the container you can reserve some of them to add later after it has reduced through evaporation.
		\item Put, uncovered, in the oven.
		\end{itemize}

	\item {\bf Initial Baking:}
		\begin{itemize}
		\item Turn the oven to 300 F.
		\item Bake for several hours (4 to 5 hours). Checking now and them.
		\item You can turn the oven off and leave the pots in the oven over night.
		\item As the meat cooks it will clump up into a large ball. Before all the liquid evaporates, you need to loosen up the meat and break all the lumps. The easiest way is to do so the next morning when the meat has cooled off. You can use your hands (with latex gloves if you wish). Otherwise, you can use a slotted metal spoon.
		\end{itemize}
		\item {\bf Second Baking:}
		\begin{itemize}
		\item Still with oven at 300 F, continue baking until all the water has evaporated from the meat and the meat has browned in several spots. 
		\item From time to time turn over the vegetables to avoid excessive browning.
		\item If you reserved some cans of tomatoes before, add them now if possible.
		\item If you see browning on the edge of the tomatoes, meat or vegetables, use a rubber spatula to loosen it and mix into the pot. The easiest way to do this is to lift a small amount of the liquid into the edges and wait about 30  seconds.
		\end{itemize}
	\item {\bf Removing Fat From the Meat}
		 \begin{itemize}
		\item Once the meat has browned some and there is no more liquid in the bottom of the pot, there will be a substantial amount of clear fat that has separated from the meat.
		\item Pull all the meat to one side of the pot and tip the pot so that the side with no meat is at a lower level.
		\item Spoon as much of the fat out as you can and put in a container for disposal (one of the empty tomato cans for example).
		\end{itemize}

	\item {\bf Assembling the Sauce and Third baking}
		\begin{itemize}
		\item Process the roasted vegetables in the food processor to obtain a paste and add to the meat.
		\item Process the roasted tomatoes in the food processor to a puree and add to the meat.
		\item It is possible that it does not feat all in one pot yet --- the sauce will continue evaporating --- keep in two pots if needed
		\item Continue baking at 300F until the sauce is very thick and you pretty much only see the meat with a heavy coat of sauce.
		\item Taste carefully to decide if you need to add additional salt.
		\item Remove the bay leaves.
		\item Put in small taperware containers, let it cool, and put in the freezer for future use.
		\end{itemize}
     	\end{enumerate}         
\end{description}
\end{document}



