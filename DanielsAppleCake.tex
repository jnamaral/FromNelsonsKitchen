\documentclass[11pt,letterpaper]{article}
\input{headings}
\newcommand \recipeName {Daniel's Apple Cake}
\chead{\recipeName}

\begin{document}
\input{title}

\begin{description}

\item[Wet Ingredients:]\  \newline
	\begin{itemize}
	\item 1/2 cup of apple sauce (or two mashed bananas)
	\item 1 1/2 cup of sugar (10 1/2 oz)
	\item 3/4 cup of canola oil ( 5 3/4 oz)
	\item 1/4 teaspoon of salt
	\item 1 teaspoon vanilla
	\end{itemize}

\item[Dry Ingredients:]\ \newline
	\begin{itemize}
	\item 3 cups of all purpose flour (15 oz)
	\item 1 teaspoon ground cinammon
	\item 1 teaspoon baking soda
	\item 1 teaspoon baking powder
	\item 3 cups of diced apples
	\item 1/2 cup of raisins
	\end{itemize}

\item[Preparation:]\ \newline
\begin{enumerate}
\item Preheat the oven to 350 F.
\item Prepare a tube pan (use baker's spray or grease it with butter - or
   shortening - and dust it with flour).
\item In a bowl mix the dry ingredients: flour, baking power, cinnamon,
   and baking soda.        
\item In a large bowl mash the bananas (or pour the apple sauce), mix
   sugar, salt, oil, and vanilla.
\item Slowly mix half of the dry ingredients with the wet ones.
\item Mix the diced apples and raisins with the other half of the dry ingredients until they are coated.
\item Dump the floured apples and raisins into the batter and mix until incorporated.
\item Pour in the prepared pan.
\item Bake for about 1 hour and 10 minutes, until a toothpick inserted in
   the center of the cake comes out clean.
\item Let it cool on a wire rack before removing from the pan.
\end{enumerate}
\end{description}
\end{document}



