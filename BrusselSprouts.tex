\documentclass[11pt,letterpaper]{article}
\input{headings}
\newcommand \recipeName {Blanched Brussel Sprouts}
\newcommand \fileName {BrusselSprouts}
\chead{\recipeName}

\begin{document}
\input{title}

This make for very tasty Brussel sprouts. I add the radishes and the grape tomatoes for interesting texture and colours. The exact amounts are not very important for this recipe. If blanching other vegetables for the same dinner, blanch the other vegetables first, blanching the Brussel sprouts and the radishes last in this order. These two vegetables leave a strong flavour in the water.
\begin{description}

\item[Ingredients:]\ \\
	\begin{itemize}
	\item 2 pounds of Brussel sprouts
	\item 1/2 pound of fresh radishes 
	\item 3/4 pound of grape tomatoes
	\item salt
	\item freshly grounded pepper
	\item 8 tablespoons of butter (or 5 tablespoons of extra-virgin olive oil)
	\item lots of ice
	\end{itemize}

\item[Equipment:]\ \\
	\begin{itemize}
	\item Large pot
	\item large bowl
	\item non-stick skillet
	\item Chinese strainer
	\item Colander
	\end{itemize}

\item[Procedure:]\ \\
	\begin{enumerate}
	\item {\bf Prepare the vegetables}
	\begin{itemize}
	\item Cut a very small slice from the stem of each Brussel sprouts and discard that slice to remove any brownish or dried spots.
	\item Slice the smaller Brussel sprouts in half and the larger ones into quarters.
	\item Wash the radishes, cut the end of the root and cut the greens and discard.
	\item Cut the small radishes in half and the larger ones into quarters and put in a separate bowl.
	\end{itemize}
	\item {\bf Set up for blanching}
	\begin{itemize}
	\item Bring water to a fast boil in a large pot that has a cover.
	\item Put cold water in a large bowl and add lots of ice to it.
	\end{itemize}
	\item {\bf Blanch the vegetables}
	\begin{itemize}
	\item Dump all the sliced Brussel sprouts into the fast boiling water and immediately cover the pot.
	\item After about 30 seconds check to see if the water came back to a boil or is very hot. If not, allow another 20 or 30 seconds.
	\item Once the water is very hot or boiling, collect the sprouts using the Chinese strainer, draining on top of the boiling pot for a few seconds and then dump into the ice cold water. Stir around the cold water to ensure that the sprouts cool off as quickly as possible.
	\item Repeat collecting the Brussel sprouts until they are all in the cold water.
	\item Cover the pot and let it come back to a full boil again.
	\item Once the Brussel sprouts are cold, remove from the cold water using the Chinese strainer and place in a colander set over a bowl to completely drain the water.
	\item Dump all the radishes into the pot of boiling water at once. 
	\item After 30 to 45 seconds remove them to the bowl of icy water.
	\item Once they are completely cold, transfer the sprouts to a bowl and transfer the radishes to the colander.
	\item Once the radishes are drained, transfer them to a bowl.
	\item Sprinkle the sprouts with a small amount of salt and let stand for at least half hour.
	\item Sprinkle the radishes with a small amount of salt and let stand for at least half an hour.
	\item Rinse the tomatoes on cold water and let they dry too.
	\end{itemize}
	\item {\bf Sautee the vegetables}
	\begin{itemize}
	\item Place a serving plater in a warming oven (at 200 F).
	\item Transfer the sprouts back to the colander to drain any water that accumulated while they were standing with the salt.
	\item Place butter into a cold non-stick skillet and put over moderate heat.
	\item Watch closely as the butter cooks, you want for the butter solids to become brown but not burned (for a vegan version, use olive oil and simply heat up the oil).
	\item Once the butter is at a light brown colour and there is no more foam on top, add all the drained sprouts. Stir for 30 seconds and then let it cook without stirring for a minute or two. The goal is to develop some golden spots in the sprouts. 
	\item Meanwhile transfer the radishes to the colander to drain any water that accumulated.
	\item Then stir and let it sit for another minute or two. Continue the process until the sprouts are half-way cooked.
	\item Add the drained radishes and the tomatoes to the skillet with the sprouts and stir.
	\item Continue stirring from time to time until the skin of some of the tomatoes start to just blister and the radishes are warmed through.
	\item Sprinkle with freshly grated black pepper.
	\item Transfer to the warmed up serving plater and serve immediately.
	\end{itemize}
	\item {\bf Optional suggestion}
	\begin{itemize}
	\item Depending of the rest of the menu, you can sprinkle the top do the dish with freshly grated parmesan cheese.
	\end{itemize}
	\end{enumerate}
\end{description}
\end{document}
