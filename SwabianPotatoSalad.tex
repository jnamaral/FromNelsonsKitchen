\documentclass[11pt,letterpaper]{article}
\input{headings}
\newcommand \recipeName {Swabian Potato Salad}
\chead{\recipeName}

\begin{document}
\input{title}

In the Summer of 2017 we travelled with my Mom through the South of Germany. It was a trip to go and visit the places from where our German ancestors had departed to go to Brazil in 1828. One Sunday afternoon we went to a Biergarten with classic music just outside of Munich called Waldwirtschaff. There I tasted this German potato salad that created a lasting palate memory. Going back home I started researching and discover that it was what is called a Swabian potato salad. Swabia is a region in the Southwest of Germany. I was fortunate to find this recipe for \href{https://www.daringgourmet.com/restaurant-style-schwabischer-kartoffelsalat-swabian-potato-salad/}{Swabian Potato Salad in a blog by Kimberly}. The blog contains step-by-step pictures that I recommend reviewing. 
 
This is an excellent salad for a party because it tastes even better on the next day and therefore it can be prepared the day before and stored in the refrigerator.

\begin{description}

\item[Ingredients:]\ \\
	\begin{itemize}
	\item 3 pounds small Yukon gold potatoes of similar size, skins scrubbed and peels left on
	\item 1 medium yellow onion, chopped
	\item 1 1/2 cups water mixed with 4 teaspoons beef bouillon granules (Vegans: use vegetable bouillon)
	\item 1/2 cup white vinegar (add a few dashes of Essig Essenz if you have it)
	\item 3/4 tablespoon salt
	\item 3/4 teaspoon freshly ground white pepper
	\item 1 teaspoon sugar
	\item 2 teaspoons mild German mustard (I recommend D�sseldorf Style German Mustard. If you can't get it, use regular yellow mustard)
	\item 1/3 cup neutral-tasting oil
	\item Fresh chopped chives for garnish
	\end{itemize}

\item[Procedure:]\ \\
	\begin{enumerate}
	\item {\bf Boil and slice the potatoes}
	\begin{itemize}
	\item Boil the potatoes in their skins in lightly salted water until tender. 
	\item Let the potatoes to cool until you can handle them. 
	\item Peel the potatoes and slice them into ? inch slices. 
	\item Put the sliced potatoes in a large mixing bowl and set aside.
	\end{itemize}
	\item {\bf Season the potatoes}
	\begin{itemize}
	\item Add onions, beef broth, vinegar, salt, pepper, sugar, and mustard to a medium saucepan and bring to a boil. 
	\item As soon as it boils, remove from heat and pour the mixture over the potatoes. 
	\item Cover the bowl of potatoes and let sit for at least one hour.
	\end{itemize}
	\item {\bf }
	\begin{itemize}
	\item After at least one hour, gently stir in the vegetable oil and season with salt and pepper to taste. 
	\item If too much liquid remains, use a slotted spoon to serve. 
	\item Serve garnished with fresh chopped chives. Serve at room temperature. 	   
	\item This potato salad is best the next day (remove from fridge at least 30 minutes before serving).
	\end{itemize}
\end{enumerate}
\end{description}
\end{document}



