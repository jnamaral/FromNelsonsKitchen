\documentclass[11pt,letterpaper]{article}
\input{headings}
\newcommand \recipeName {Braised Fennel}
\chead{\recipeName}

\begin{document}
\input{title}

\begin{flushright}
{\hspace{4in} From Marcella Hazan.}
\end{flushright}
\vspace{0.5in}



\begin{description}

\item[Ingredients:]\ \\
	\begin{itemize}
	\item	3 large fennel bulbs or 4 to 5 smaller ones
	\item 1/3 cup of extra virgin olive oil
	\item Salt
	\end{itemize}

\item[Procedure:]\ \\

	\begin{enumerate}
	\item {\bf Preparing the Fennel}
	\begin{itemize}
	\item Cut the fennel tops where they meet the bulb and discard them.
	\item Detach and discar any of the bulb's outer parts that may be bruised or discoloured.
	\item Slice 1/8 inch of the butt end.
	\item Cut the bulb vertically into slices about 1/3 inch thick.
	\end{itemize}

	\item {\bf Braising the Fennel}
	\begin{itemize}
	\item Put the fennel and the olive oil in a large saucepan.
	\item Sprinkle with salt.
	\item Add enough water to barely cover the fennel.
	\item Turn heat to medium.
	\item Do not put a lead in on the pot.
	\item Cook, turning slices over from time to time until the fennel is glossy and pale gold and it is tender when pierced with a knife --- between 25 and 40 minutes.
	\item If the liquid is insufficient, add a bit more water.
	\item All the water must be evaporated by the time the fennel is cooked.
	\item Serve on a warm plate.
	\end{itemize}

	\end{enumerate}
\end{description}
\end{document}



