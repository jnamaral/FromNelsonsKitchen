\documentclass[11pt,letterpaper]{article}
\input{headings}
\newcommand \recipeName {Tarta de Santiago}
\newcommand \fileName {TartaDeSantiago}

\chead{\recipeName}

\begin{document}
\input{title}

%\begin{flushright}
%{\bf From {\it } by }
%\end{flushright}
 
I first had this cake in Santiago de Compostela. It is so traditional and delicious that several shops sell it in boxes for tourists to take it home. There are several recipes in the internet. This one is a combination of a recipe published by Milk Street and one from Spanish Sabores.
\begin{description}

\item[Ingredients:]\ \\
	\begin{itemize}
	\item zest from half a lemon
	\item 250 grams of sugar
	\item 5 eggs
	\item 3/4 teaspoon of almond extract
	\item 1/4 teaspoon of salt 
	\item 1/4 teaspoon of vanilla
	\item 250 grams of blanched almond flour
	\item butter for greasing the pan
	\item powdered sugar for dusting
	\end{itemize}

\item[Procedure:]\ \\
	\begin{enumerate}
	\item {\bf Prep the cake pan and warm the oven}
	\begin{itemize}
	\item Pre-heat the oven to 350F.
	\item Cut a piece of parchment paper that is large enough to cover the bottom of a 10 inch springform pan.
	\item Fold the paper in half, and then in half again, bring edges together to form a triangle. Keep folding until you have a thin long triangle.
	\item Place on top of the springform pan with the point on the centre and cut at the edge of the pan with a scissor to make a circle the size of the bottom of the pan.
	\item Put the paper circle on the bottom of the pan and butter the paper and the sides of the pan.
	\end{itemize}
	\item {\bf Mix the cake }
	\begin{itemize}
	\item Put sugar, salt, and lemon zest on the bowl of a standing mixer and process for 30 seconds.
        \item Add the eggs, almond extract and vanilla and beat until the mixture is pale and creamy.
	\item Fold the almond flour with a rubber spatula.
	\end{itemize}
	\item {\bf Bake the cake}
	\begin{itemize}
	\item Pour the batter on the prepared pan
	\item Bake for 40 to 50 minutes until a toothpick inserted in the middle of the cake comes out almost clean.
	\end{itemize}
	\item {\bf Remove from pan and cool}
	\begin{itemize}
	\item Let it cool for about 10 minutes in the baking pan.
	\item Run a knife around the edges of the cake and remove the sides by opening the spring.
	\item Place a light clean baking sheet on top of the cake so that the underside of the baking sheet is touching the cake.
	\item Turn the cake upside down on top of the bottom of the baking sheet.
	\item Remove the baking pan bottom.
	\item Remove the circle of parchment paper.
	\item Place a wire rack on top of the inverted cake and invert it again so that the cake is sitting top side up on top of the wire rack.
	\item Let it cool until it is at room temperature or just lukewarm.
	\end{itemize}
	\item {\bf Decorate}
	\begin{itemize}
	\item Print from the internet a Cross of Saint James on sturdy paper and cut a stencil of the cross.
	\item Place the stencil on top of the cake.
	\item Dust the cake liberally with confectioner sugar.
	\item Carefully remove the stencil.
	\end{itemize}
	\end{enumerate}

\end{description}
\end{document}



