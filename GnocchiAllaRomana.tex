\documentclass [11pt, letterpaper] {article}
\input {headings}
\newcommand \recipeName {Gnocchi alla Romana}
\newcommand \fileName {GnocchiAllaRomana}
\chead {\recipeName}

\begin {document}
\input {title}

\begin{flushright}
{From Mario Batali.}
\end{flushright}

We were in Rome for a vacation, our last couple's vacation before Daniel came. We went on a food tour of Rome and met a New Yorker food critic living in Rome. She gave us a list of small hard-to-find restaurants that we must visit. In one such restaurant in a narrow alley, I tried for the first time this dish. I was hooked right away. Back home I searched for it and I found this recipe from Mario Batali. It works very well. It is one of those very reach dishes that tastes "light".

\begin{description}

\item[Ingredients:]\ \\
	\begin{itemize}
	\item	3 cups  (2 lb + 1 1/2 oz) of whole milk
	\item 1 teaspoon of salt
	\item 6 tablespoons of butter plus 2 tablespooons
	\item 1 cup (5 1/2 oz) of semolina
	\item 1/2 cup (2 oz) of grated Parmigianno-Reggiano plus 1/2 cup (2 oz)
	\item 4 egg yolks
	\end{itemize}

\item[Procedure:]\ \\

	\begin{enumerate}
	\item {\bf Prepare Surface}
	\begin{itemize}
	\item Clean an area in a countertop or use a baking sheet. 
	\item Spray with cooking spray.
	\end{itemize}
	
	\item {\bf Cook Gnocchi }
	\begin{itemize}
	\item In a large non-reactive saucepan heat up the milk, butter and salt.
	\item Add semolina in a steady thin stream while whisking vigorously with a wire whisk.
	\item As the mixture thicken, switch to a flat wooden spoon. 
	\item Cook until mixture is thickened and starts loosening from the bottom of the pan.
	\end{itemize}

	\item {\bf Incorporate Yolks and Parmesan}
	\begin{itemize}
	\item Remove from heat.
	\item Incorporate egg yolks mixing vigorously.
	\item Incorporate 1/2 cup of grated parmesan.
	\item Pour on prepared surface and spread to 1/2 inch thickness.
	\item Allow to cool.
	\end{itemize}

	\item {\bf Baking the Gnocchi}
	\begin{itemize}
	\item Preheat the oven to 425 F.
	\item Grease a baking dish with butter.
	\item Cut the Gnnochi into small squares.
	\item Arrange the squares on the baking dish.
	\item Sprinkle with the remaining 1/2 cup of parmesan cheese.
	\item Bake until the top is lightly brown.
	\item Serve immediately.
	\end{itemize}

	\end{enumerate}
\end{description}
\begin{table}
\begin{tabular}{cccc}
\includegraphics[width=0.25\textwidth]{\imageDir/\fileName/IMG_3197.jpg} &
\includegraphics[width=0.25\textwidth]{\imageDir/\fileName/IMG_3212.jpg} &
\includegraphics[width=0.25\textwidth]{\imageDir/\fileName/IMG_3213.jpg} \\
\includegraphics[width=0.25\textwidth]{\imageDir/\fileName/IMG_3206.jpg} &
\includegraphics[width=0.25\textwidth]{\imageDir/\fileName/IMG_3214.jpg} &
\includegraphics[width=0.25\textwidth]{\imageDir/\fileName/IMG_3216.jpg} \\
\includegraphics[width=0.25\textwidth]{\imageDir/\fileName/IMG_3217.jpg} &
\includegraphics[width=0.25\textwidth]{\imageDir/\fileName/IMG_3218.jpg} &
\includegraphics[width=0.25\textwidth]{\imageDir/\fileName/IMG_3219.jpg} \\
\includegraphics[width=0.25\textwidth]{\imageDir/\fileName/IMG_3220.jpg} &
\includegraphics[width=0.25\textwidth]{\imageDir/\fileName/IMG_3228.jpg} \\
\end{tabular}
\end{table}

\end{document}



