\documentclass[11pt,letterpaper]{article}
\input{headings}
\newcommand \recipeName {Blanched Vegetables}
\newcommand \fileName {BlanchedVegetables}
\chead{\recipeName}

\begin{document}
\input{title}

The exact amounts are not very important for this recipe. Blanching not only preserves the colours in the vegetables, but also gives a better flavour for the final dish

\begin{description}

\item[Ingredients:]\ \\
	\begin{itemize}
	\item 8 small sweet peppers
	\item 2 or 3 sticks of celery
	\item 3 carrots
	\item 1 fennel bulb
	\item 1 cup of frozen peas
	\item salt
	\item freshly grounded pepper
	\item 3 to 4 tablespoons  of olive oil
	\item lots of ice
	\end{itemize}

\item[Equipment:]\ \\
	\begin{itemize}
	\item Large pot
	\item large bowl
	\item non-stick skillet
	\item Chinese strainer
	\item Colander
	\end{itemize}

\item[Procedure:]\ \\
	\begin{enumerate}
	\item {\bf Prepare the vegetables}
	\begin{itemize}
	\item Remove seeds and stems from the peppers and cut into small strips put in a bowl.
	\item Remove strings from celery ribs by lifting from the end of each rib and pulling. Cut the ribs on a diagonal into fine slices.
	\item Peel the carrots and cut into small wedges by rotating the carrot about a quart of a turn every time you cut it.
	\item Cut the fennel bulb in half, remove the hard core at the bottom. Then slice the fennel in the thin set of a mandolin. It will produce large slices. Using a chefs knife cut the slices into about three parts to make them smaller.
	\end{itemize}
	\item {\bf Set up for blanching}
	\begin{itemize}
	\item Bring water to a fast boil in a large pot that has a cover.
	\item Put cold water in a large bowl and add lots of ice to it.
	\end{itemize}
	\item {\bf Blanch the vegetables}
	\begin{itemize}
	\item Dump all the sliced fennel into the fast boiling water and immediately cover the pot.
	\item After about 30 seconds check to see if the water came back to a boil or is very hot. If not, allow another 20 or 30 seconds.
	\item Once the water is very hot or boiling, collect the fennel using the Chinese strainer, draining on top of the boiling pot for a few seconds and then dump into the ice cold water. Stir around the cold water to ensure that the sprouts cool off as quickly as possible.
	\item Repeat collecting the fennel until they are all in the cold water.
	\item Cover the pot and let it come back to a full boil again.
	\item Once the fennel are cold, remove from the cold water using the Chinese strainer and place in a colander set over a bowl to completely drain the water.
	\item Transfer the drained fennel to a bowl and sprinkle with salt and let stand for at least 30 minutes.
	\item Repeat the process with the carrots, then the celery, and then the peppers.
	\end{itemize}
	\item {\bf Sautee the vegetables}
	\begin{itemize}
	\item Place a serving plater in a warming oven (at 200 F).
	\item Dump each of the vegetables over a strainer set on top of a bowl to drain any water that accumulated while they were standing with the salt.
	\item Place olive oil into a cold non-stick skillet and put over moderate heat.
	\item Once the oil is shimmering, add the frozen peas to the oil and let it cook for a minute or so without stirring.
	\item Add all other vegetables to the peas and stir fry until they are all warmed through.
	\item Sprinkle with freshly grated black pepper.
	\item Transfer to the warmed up serving plater and serve immediately.
	\end{itemize}
	\end{enumerate}
\end{description}
\end{document}
