\documentclass[11pt,letterpaper]{article}
\input{headings}
\newcommand \recipeName {My Mom's Pork Medallions}
\chead{\recipeName}

\begin {document}
\input {title}

\begin{flushright}
This recipe is translated, with a few adaptations, from my Mom's, Dioraci Rambo Urtassum, cookbook  {\it Delícias, aromas e vidas}.
\end{flushright}

\begin{description}

\item[Ingredients:]\ \\
	\begin{itemize}
	\item 2 lb(s) of pork tenderloin
	\item 2 teaspoons of salt
	\item 1/2 teaspoons of ground cinnamon
	\item 1/4 teaspoon of ground cloves
	\item 1 clove of garlic
	\item 1 pimenta de cheiro or 4 droplets of red pepper sauce
	\item 1 tablespoon of soy sauce
	\item 1/2 cup of water
	\item 4 tablespoons of butter
	\item 2 tablespoons of cooking oil
	\item 2 tablespoons of rosemary minced
	\item 1 tablespoon of corn starch
	\end{itemize}

\item[Procedure:]\ \\
        \begin{description}
        \item[Marinate the Pork]\ \\
                \begin{itemize}
			\item Trim the tenderloins by removing the silver skin and all the fat.
			\item Cut the thin ends of the tenderloins and reserve for another use.
			\item Slice the tenderloins in 1 1/2 to 2 inch thick slices.
			\item Smash the garlic with the salt using a chef's knife on a wooden cutting board to obtain a paste.
			\item If using the pimenta the cheiro, also smash the pepper into the garlic paste.
			\item Otherwise add the drops of hot pepper sauce to the garlic paste and stir well to incorporate.
			\item Rub the garlic paste into the pieces of pork tenderloin.
			\item Mix the ground cinnamon with the ground cloves and pulverize over the pork.
			\item Add the 1/2 cup of water, cover, and put in the refrigerator for at least one hour. This can be done several hours ahead of time.
		\end{itemize}
	\item[Air Dry the Pork]\ \\
		\begin{itemize}
			\item Set a wire rack over a baking sheet.
			\item Squash each piece of pork with the palm of your hand against a cutting board to create thiner and wider medallions.
			\item Place each medallion on top of the rack leaving some space between then. 
			\item Place, uncovered in the refrigerator for at least a couple of hours to get the pieces of pork to dry some. This way they will brown better later.
		\end{itemize}
	\item[Brown the Pork]\ \\
		\begin{itemize}
			\item Choose a heavy-bottom pan with a fitting cover. I use a stainless steel one. Add 2 tablespoons of the butter and the cooking oil and heat it up until the butter has melted and is starting to color.
			\item Place pieces of pork in the pan, without crowding it. Do not move the pieces after you place them. Let them cook under high heat until the first side is browned. 
			\item Turn the pieces and brown the second side.
			\item Be careful to not overcook. The medallions should still have a ring of very pink raw meat in the middle.
			\item If needed, brown the pieces of pork in batches, keeping them in a warmed covered plate once their are brown.
			\item Once you are done with all pieces of pork, put all of them in the pan, remove the pan from the hot burner into a cold one, cover tightly and let seat, covered, for thirsty minutes.
			\item Meanwhile, put your serving plate in an oven at 200 F to warm it up.
		\end{itemize}	
	\item [Finish the Sauce and Serve]\ \\
		\begin{itemize}
			\item Mix the cornstarch with two tablespoons of water in a small bowl or cup until dissolved.
			\item Remove the pork medallions to the serving plate, cover, and keep it warm.
			\item Add the soy sauce to the pan.
			\item Scrape all brown bits from the pan into the sauce.
			\item Add the dissolved corn start to the pan.
			\item Place the pan over moderately high heat and cook stirring until the sauce thickens.
			\item Remove from the stove and add the two tablespoons of butter and the minced rosemary, stirring constantly until the butter dissolves.
			\item Pour the sauce over the pork medallions and serve.
		\end{itemize}
	\end{description}
\end{description}
\end{document}
