\documentclass[11pt,letterpaper]{article}
\input{headings}
\newcommand \recipeName {Julia Child's Pastry Dough}
\chead{\recipeName}

\begin{document}
\input{title}

\begin{flushright}
From {\it The Way to Cook} by Julia Child
\end{flushright}

This is a master recipe. The details are important. The most important thing is to not overwork this dough and not add too much water. The dough will just come together and you will still see small pieces of butter in it.
 
\begin{description}

\item[Ingredients:]\ \\
	\begin{itemize}
	\item 1 1/2	cups unbleached flour
	\item 1/2	cup cake flour
	\item 1	teaspoon salt
	\item 6	ounces cold cultured unsalted butter  (12 tbs or 1 1/2 sticks)
	\item 4	tablespoons vegetable shortening or leaf lard (I like to use lard)
	\item 1/2	cup ice water
	\end{itemize}

\item[Procedure:]\ \\
	\begin{enumerate}
	\item {\bf Cut the Butter}
	\begin{itemize}
	\item Using a paring knife, cut the cold butter into 1/2 inch cubes.
	\end{itemize}
	\item {\bf Food Processor Method}
	\begin{itemize}
	\item Add the flour and salt pulse once to mix.
	\item Add the butter pulse 5 or 6 times to break up the butter.
	\item Add the shortening or lard and pulse until it resembles crumbs.
	\item Add the ice water pulsing until you have a cohesive dough, it should hold together when you press it in your hand. Do not overwork the dough.
	\end{itemize}
	\item {\bf Pastry Cutter Method}
	\begin{itemize}
	\item Add the flour and salt and stir with a fork.
	\item Add the butter and break it up with your pastry cutter, a fork, or the tip of your fingers.
	\item Add the shortening or lard and mix it until it resembles crumbs. Yu should still see little pieces of butter in the dough.
	\item Add the water a little at a time until you have a cohesive dough that holds together when pressed in your hand. 
	\end{itemize}
	\item {\bf Form and Wrap the Dough}
	\begin{itemize}
	\item Dump out onto work surface and form into a large square (or two discs if you will use for pies). 
	\item Wrap in plastic and refrigerate at least 30 minutes but one hour is preferable. 
	\item The dough can be kept refrigerated for several days.	
	\end{itemize}
	\end{enumerate}
\end{description}
\end{document}



