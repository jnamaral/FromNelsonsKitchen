
\documentclass[11pt,letterpaper]{article}
\input{headings}
\newcommand \recipeName {Shrimp Bob\'ozinho}
\chead{\recipeName}

\begin{document}
\input{title}


I have liked a dish called {\it Shrimp Bob\'o} since I first learned about it in a trip to Bahia in the 1990s. It is a delicious hearty soup with a base made with onions, tomatoes, hot pepper, coconut milk, thickened with cooked mandioca and flavoured with the heads and shells of shrimp. The shrimp themselves are added at the last minute. A Bob\'o has a gravy thickened with the starch from the cooked mandioca, but pieces of mandioca are still visible in the gravy alongside the barely cooked shrimp.

It was in a visit to a very special restaurant called Tordesilhas in S\~ao Paulo in 2022 that I was introduced to the concept of a {\it Shrimp Bob\'ozinho}. In Portuguese, the diminutive suffix {\it inho} makes whatever it is modifying small. The idea is to make a more sophisticated version of this dish and serve as an appetizer either in cups or in small bowls. The flavours are essentially the same, but this preparation and presentation turns it into a sophisticated appetizer or first course.

This is a simple recipe, but there are several details that will affect the final product. Preparing the shrimp broth, and being careful about how you handle the coconut milk, the cilantro, the shrimp, and the dend\^e oil at the end are essential because of the volatile flavours of these two products.

You can buy frozen shrimps, thaw them, and prepare as specified in the recipe. In Edmonton I buy peeled frozen mandioca at Tienda Latina. Make sure to pick bags where the mandioca does not have much freeze burning. They also have dend\^e oil.
 
 
\vspace{0.3in}

\begin{description}

\item[Ingredients:]\ \\
	\begin{itemize}
	\item 1 1/2 lb of shrimp with heads and shell.
	\item 2 cups of cooked mandioca
	\item 1/2 cup of canned tomatoes
	\item 1 large shallot diced
	\item 2 tablespoon of cooking oil
	\item 1 or 2 pimenta de cheiro (or a few drops of hot sauce such as Tabasco)
	\item 1 small can of coconut milk
	\item 2 tablespoon of cilantro
	\item Salt to taste
	\item 1 tablespoon of dend\^e oil (optional)
	\end{itemize}

\item[Procedure:]\ \\
	\begin{enumerate}
	\item {\bf Prepare a shrimp broth}
	\begin{itemize}
	\item Remove the heads and the shell of the shirimp.
	\item Heat one tablespoon of cooking oil in a pan.
	\item When the pan is very hot, add all the shrimp heads and shells at once and stir for a few minutes until they all turn pink.
	\item Add two cups of water to the pan and let it boil for a few minutes for the water to absorve the flavour of the shrimp heads and shell.
	\item Strain the broth and reserve. Discard the shells and heads.
	\item The broth can be prepared a day ahead and kept in the fridge or several days ahead and frozen.
	\end{itemize}
	\item {\bf Devein  and dice the shrimps}
	\begin{itemize}
	\item With a small sharp paring knife cut a slit all the way along the back of each shrimp.
	\item Remove the dark intestinal track. Use a paper towel to collect the intestinal tracks and later discard them.
	\item Dry the shrimps with paper towels.
	\item Cut the shrimps. One elegant way is to slice each shrimp in half lengthwise and them, depending on their size, cut each half into two or three pieces so that one can still recognize the shrimps in the final dish.
	\item Put the diced shrimp in a closed container in the refrigerator.
	\item If preparing the broth a few days ahead, freeze the shrimp and make sure to thaw in the fridge or at room temperature before proceeding.
	\end{itemize}
	\item {\bf Prepare the Cooked Mandioca}
	\begin{itemize}
	\item Place the peeled frozen mandioca in a pressure cooker and cover with plenty of water.
	\item Close the pressure cooker and bring to high pressure.
	\item Cook at high pressure for thirty minutes.
	\item Turn of the stove and let it cool down until the pressure cooker can be open.
	\item Add salt to the water and let the mandioca soak in the salted water for at least half an hour.
	\item You likely will have more cooked mandioca than you need for this recipe. You can use it for other dishes. Brazilians love to deep fry the cooked mandioca. In that case it is best to place the cooked mandioca on a wire rack over a baking sheet and let it dry in the fridge overnight before frying.
	\item Pick the softest pieces of mandioca to add to this recipe. You can dice it into smaller pieces to make processing easier later. You will need about two cups.
	\item You can also use the mandioca broth to thin the Bob\'ozinho at the end if needed.
	\end{itemize}

	\item {\bf Prepare the Base for the Bob\'o}
	\begin{itemize}
	\item Heat one tablespoon of oil in a medium saucepan.
	\item Sweat the shallots until their are translucent, but do not brown them.
	\item Add the tomatoes and cook for a couple of minutes.
	\item Add the shrimp broth and the cooked mandioca.
	\item Mince the pimenta de cheiro and add to the pot --- alternatively add a few drops of hot sauce.
	\item Let the broth cook in gentle heat for about ten minutes.
	\item Remove from the heat.
	\item Add the cilantro -- you can opt to use only cilantro stems in this recipe.
	\item Using an immersion blender process the soup until it is completely smooth. Alternativelty
	\item The recipe can be prepared up to this point ahead of time. Make sure to warm it completely before proceeding.
	\end{itemize}
	\item {\bf The Coconut Milk and the Shrimp --- at Serving Time}
	\begin{itemize}
	\item Pour the coconut milk in a microwave-safe dish and warm it in the microwave until it is very hot but not boiling.
	\item Off the head, add the hot coconut milk to the hot soup.
	\item If the soup is thick, add some hot mandioca broth to thin it.
	\item Add the diced shrimp, stir, cover the pot and let sit for two or three minutes until the shrimp just barely turn pink.
	\item If using, stir in the dend\^e oil at the last minute just before serving.
	\item Ladle the Bob\'ozinho into cups or small bowls making sure that each has a few pieces of shrimp. 
	\item Pass around providing a spoon to each guest.
	\end{itemize}
	\end{enumerate}
\end{description}
\end{document}



