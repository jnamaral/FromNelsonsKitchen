\documentclass[11pt,letterpaper]{article}
\input{headings}
\newcommand \recipeName {Couscous with Dates and Almonds}
\chead{\recipeName}

\begin{document}
\input{title}

Cooking couscous in the traditional Moroccan way is very laborious. The North-American way of cooking it (add to hot liquid and let stand) produces unsatisfying result. This recipe is a good compromise and it cooks the couscous in the pilaf style, which is the way Brazilians cook rice.
This is an adaptation of a recipe for {\it Couscous with Dates and Pistachios} from the America's Test Kitchen. Besides replacing the pistachios for almonds, I also reserve the dates to be added in the end so that they keep their profile.

\begin{description}

\item[Ingredients:]\ \\
	\begin{itemize}
	\item 3 tablespoons unsalted butter
	\item 2 cups couscous
	\item 1 tablespoon finely grated fresh ginger
	\item 1/2 teaspoon ground cardamom
	\item 1 1/4 cups water
	\item 1 cup low-sodium chicken broth
	\item 1 teaspoon table salt
	\item 3/4 cup coarsely toasted almonds, chopped roughly
	\item 3 tablespoons minced fresh cilantro leaves
	\item 2 teaspoons lemon juice
	\item 1/2 cup chopped dates
	\item Ground black pepper
	\end{itemize}

\item[Procedure:]\ \\
	\begin{itemize}
	\item Heat butter in medium saucepan over medium-high heat until foaming subsides.
	\item Add couscous, ginger, cardamom and cook, stirring frequently, until grains are just beginning to brown, about 5 minutes. 
	\item Add water, broth, and salt; stir briefly to combine, cover, and remove pan from heat. 
	\item Let stand until grains are tender, about 7 minutes. 
	\item Uncover and fluff grains with fork. 
	\item Stir in cilantro and lemon juice, season with pepper to taste.
	\item Add the pistachios e dates, stir, and serve.
	\end{itemize}
\end{description}
\end{document}
