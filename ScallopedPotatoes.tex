\documentclass[11pt,letterpaper]{article}
\input{headings}
\newcommand \recipeName {Creamy Scalloped Potatoes}
\newcommand \fileName {ScallopedPotatoes}
\chead{\recipeName}

\begin{document}
\input{title}

Rather than a super-rich dish with heavy cream and cheese, I prefer a lighter version of scalloped potatoes. This recipe is inspired by one in Julia Child's {\it The Way to Cook} cookbook. I have tried to make this dish by simply replacing the heavy cream with whole milk. The problem is that the raw potato has an enzime that makes the milk curdle leading to an unappealing dish. The solution is to make a very light bechamel sauce first. The bonding of the milk with the butter and the flour prevents the curdling of the milk. 

\begin{description}

\item[Ingredients:]\ \\
	\begin{itemize}
	\item 2 1/2 pounds of Yukon gold or white potatoes
	\item 6 cups of milk
	\item 5 Tablespoons (66 grams) of butter (more to butter the baking dish)
	\item 7 Tablespoons (50 grams) of flour 
	\item 2 teaspoons of salt
	\item 1/2 teaspoon of white pepper
	\item 1 clove of garlic
	\end{itemize}

\item[Optional Ingredients:]\ \\
	\begin{itemize}
	\item 1 1/2 Tablespoon of finely chopped chives
	\item 1 1/2 Tablespoon of finely chopped fresh sage
	\end{itemize}

\item[Procedure:]\ \\
	\begin{enumerate}
	\item {\bf Prepare the baking dish and pre-heat the oven.)}
		\begin{itemize}
		\item  Butter a baking dish.
		\item Preheat the oven to 375 F.
		\end{itemize}
	\item {\bf Make the Bechamel Sauce}
		\begin{itemize}
		\item Measure the milk in a large measuring glass dish.
		\item Heat the milk in a microwave.
		\item Meanwhile, melt the butter over moderate heat.
		\item Once the butter is melted, remove the pan from the stove, add the flour to the melted butter and stir with a whisk balloon. There is no need to cook the butter and flour and keeping it cooler will make it easier to produce a smooth sauce without lumps. 
		\item Remove from the fire and gently add the hot milk stirring vigorously in the beginning to prevent lumps from forming. 
		\item Once all the hot milk has been added, switch to a wood spoon. Cook, stirring, for a few minutes until the sauce thickens just slightly. Make sure that the sauce is boiling. It will be a thin sauce.
		\end{itemize}
	\item {\bf Season the sauce}
		\begin{itemize}
        		\item On a wood cutting board, using a chef's knive, crush the garlic. Then add the salt to the garlic and continue crushing until you obtain a paste.
		\item Add the white pepper to the paste and crush again.
		\item Add this paste to the warm sauce and stir well.
		\item If using, chop the herbs and add to the warm sauce. 	
		\end{itemize}
	\item {\bf Slice the potatoes and add to the sauce}
		\begin{itemize}
        		\item Using a food processor, slice the potatoes in 2.5 mm thick slices.
		\item Spoon a small amount of sauce on the bottom of the baking dish.
		\item Dump all the sliced potatoes into the hot sauce, put back on top of the stove, and cook, stirring gently with a wooden spoon to not break the potatoes. You want to ensure that there are no stacks of potato slices without sauce in between the slices.
		\item Dump the potato and sauce mixture into the prepared dish and level the top. It will be very saucy.
		\item Spray a sheet of aluminum foil with cooking spray and cove the dish.
		\item Bake cover for about 45 minutes --- a small sharp knife must easily pierce the potatoes slices on the center of the dish.
		\item Remove the foil and continue baking for another 20 minutes.
		\item Turn the oven to broil and watch the dish, remove when the top develops a few brown spots.
		\item Remove from oven and let is cool, uncovered, for about 15 minutes before serving.
		\end{itemize}
	\item{\bf Brown the top}
		\begin{itemize}
        		\item Increase the oven temperature to 400 F.
		\item Remove the aluminum foil from the top of the dish and return the dish to the oven.
		\item Bake for about 10 minutes until the top of the dish starts browning slightly.
		\item Remove from the oven, and let it cool for ten or fifteen minutes before serving.	
		\end{itemize} 	
     	\end{enumerate}         
\end{description}
\begin{table}
\begin{tabular}{cccc}
\includegraphics[width=0.25\textwidth]{\imageDir/\fileName/IMG_3197.jpg} &
\includegraphics[width=0.25\textwidth]{\imageDir/\fileName/IMG_3212.jpg} &
\includegraphics[width=0.25\textwidth]{\imageDir/\fileName/IMG_3213.jpg} \\
\includegraphics[width=0.25\textwidth]{\imageDir/\fileName/IMG_3206.jpg} &
\includegraphics[width=0.25\textwidth]{\imageDir/\fileName/IMG_3214.jpg} &
\includegraphics[width=0.25\textwidth]{\imageDir/\fileName/IMG_3216.jpg} \\
\includegraphics[width=0.25\textwidth]{\imageDir/\fileName/IMG_3217.jpg} &
\includegraphics[width=0.25\textwidth]{\imageDir/\fileName/IMG_3218.jpg} &
\includegraphics[width=0.25\textwidth]{\imageDir/\fileName/IMG_3219.jpg} \\
\includegraphics[width=0.25\textwidth]{\imageDir/\fileName/IMG_3220.jpg} &
\includegraphics[width=0.25\textwidth]{\imageDir/\fileName/IMG_3228.jpg} \\
\end{tabular}
\end{table}

\end{document}



