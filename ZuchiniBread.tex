\documentclass[11pt,letterpaper]{article}
\input{headings}
\newcommand \recipeName {Zuchini Bread with Prunes}
\chead{\recipeName}

\begin{document}
\input{title}

This recipe gets to this collection through a tortuous route. In the early 1990s I was visiting Scott's aunt Diane in Tulsa, Oklahoma, and someone gave me this recipe cut out of a Sunday newspaper. The main feature was that it has no fat and no eggs. I baked this bread many times. When moving to Brazil I adapted it to use coyote squash, much more readily available there, and passed the recipe to my friend Gloria, who lives in Santos, SP. I believe that Gloria was visiting in Porto Alegre of Xangril\'a, RS, when I baked the bread and she liked it very much. Then, years later, when we moved out of a house in Newark, DE, we forgot my folder of recipes and the landlady discarded it. I thought that I had lost the recipe and thought about it now and then. In 2017 I was at the airport in Frankfurt, Germany, when I received a text message from Gloria with a picture of the recipe saying, ``Look what I found in a drawer! Great memories." Thus I recovered the recipe. 

I like this bread very much because I like the flavour of prunes. However, this is not a popular bread in my house where others do not share this taste for prunes. I find it curious that taste for prunes is surprisingly dividing amongst people. If it is zuchini season you can make several loafs and store in the freezer soon after baking.


\begin{description}

\item[Ingredients:]\ \\
	\begin{itemize}
	\item	1 cup of grated zuchini
	\item 1 1/2 cup of sugar (10 1/2 oz)
	\item 1 cup of water
	\item 1/2 cup of prune puree
	\item 2 teaspoons of lemon zest
	\item 1/2 Tablespoon of lemon juice
	\item 1 teaspoon of vanilla
	\item 1 teaspoon of salt
	\item 1 1/2 teaspoon of cinnamon
	\item 1 teaspoon of baking soda
	\item 1 teaspoon of baking powder
	\item 2 1/2 cups of flour (12 1/2 oz)
	\item 3/4 cup of diced prunes
	\item 1/2 cup of chopped nuts
	\end{itemize}

\item[Procedure:]\ \\

	\begin{enumerate}
	\item {\bf Preheat oven and prepare the pan}
	\begin{itemize}
	\item Preheat oven to 325 F.
	\item Cut a rectangular piece of parchment paper that covers the bottom of the loaf pan and that come up the long side of the pan.
	\item Spray the bottom of the pan with cooking spray.
	\item Line the pan with the parchment paper and spray the parchment paper with the cooking spray.
	\end{itemize}
	
	\item {\bf Preparing the prune puree}
	\begin{itemize}
	\item The best is to make a larger batch of the prune puree. If you are making a single recipe of the bread, you can store the remaining puree in small containers in the refrigerator or in the freezer.
	\item Dice prunes and mix with a small amount of water and process in the blender or in a small food processor until you obtain a thick paste.
	\item If making a single batch you can slowly add the 1 cup of water that the recipe calls for to the blender to loosen up the paste and make it easier to remove from the blender/processor. However do not add too much water too soon because the mixture must be thick to enable the breakdown of the prunes into a paste.
	\end{itemize}

	\item {\bf Mix the wet ingredients}
	\begin{itemize}
	\item In a large bowl, mix the grated zuchini with the water, sugar, lemon zest, lemon juice, vanilla, salt, and prune puree.
	\end{itemize}
	
	\item {\bf Dump dry ingredients, then mix without stirring}
	\begin{itemize}
	\item Dump all the flour in a mound on top of the wet ingredients without mixing.
	\item Spread the cinnamon, baking power, baking soda on top of the flour.
	\item With a fork or a dry whisk gently mix only the dry ingredients.
	\end{itemize}
	
	\item {\bf Flour diced prunes and chopped nuts, then stir}
	\begin{itemize}
	\item Spread the diced prunes and the chopped nuts on top of the dry ingredients and gently move them around. The goal is to coat the prunes and nuts with the dry ingredients before mixing the batter. This step helps the nuts and prunes to remain distributed through the batter.
	\item Now, using a large rubber spatula or a wooden spoon, gently stir the batter together until it is fairly homogeneous but do not overmix.
	\end{itemize}
		
	\item {\bf Bake, cool and serve}
	\begin{itemize}
	\item Pour the batter in the prepared loaf pan.
	\item Bake at 325 F for about one hour or until a toothpick comes out clean when inserted in the middle of the loaf. 
	\item Remove from oven and let it cool, in the pan,  on a rack for 10 minutes.
	\item Run a sharp knife along the short ends of the pan that were not covered by the parchment paper to release the bread.
	\item Invert on top of a cutting board.
	\item Peel the parchment paper from the bread.
	\item Invert again (you want the top side up now) on top of the cooling rack. 
	\item Serve warm or let cool to room temperature.
	\item If it is completely cool when you serve, you may warm up a slice for 20 seconds in full power in the microwave. Best is to have a plastic cover over the slice. 
	\item Serve warm with a spoonful of sour cream.
	\end{itemize}
	
	\end{enumerate}
\end{description}
\end{document}



