\documentclass[11pt,letterpaper]{article}
\input{headings}
\newcommand \recipeName {Bife \`a Milanesa}
\chead{\recipeName}

\begin{document}
\input{title}

{\em Bife \`a Milanesa} is a dish that says ``home" very loudly to me. This has been my favourite dish since I was a child. The quality and the freshness of the beef and the eggs matters for this dish. When picking the minute steaks at the store avoid anyone with brownish or greenish spots. It is also better if the steaks were cut ``against the grain'', {\em i.e.} in a way that sections long muscle fibres. If you are able to, cut your own minute steaks from a lean piece of beef, such as a sirloin. You can slice the minute steaks and season them in advance. This dish is best if served as soon as it is finished. But it also tastes delicious as a cold snack or in a sandwich the next day. If preparing a large number of minute steaks for a crowd, you can place them on a rack over a baking sheet in a 200 F oven to keep them warm as you cook them in batches.

I speculate that this dish is an influence of the Italians that migrate to the South of Brazil in the mid 1870s.

\vspace{.3in}

\begin{description}

\item[Ingredients:]\ \\
	\begin{itemize}
	\item 6 minute steaks
	\item 2 (or 3) large eggs
	\item bread crumbs
	\item salt
	\item finely ground pepper (black or white)
	\item canola oil
	\item butter
	\item 1 tablespoon of finely chopped parsley
	\end{itemize}

\item[Procedure:]\ \\
	\begin{enumerate}
	\item {\bf Prep the steaks}
	\begin{itemize}
	\item If the steaks are very large, cut them in two or three pieces to make them easier to sautee.
	\item Pound each steak individually with a meat mallet.
	\item Mix a small amount of ground white pepper in the salt.
	\item Sprinkle a bit of the salt and pepper mixture in each stake, piling the next steak on top of the previous one until you are done.
	\end{itemize}
	\item {\bf Prepare the egg wash}
	\begin{itemize}
	\item Choose platters that are large enough to enable you to work comfortably with the steaks. In one platter spread dry, finely ground, toasted bread crumbs --- for best results, toast fresh bread to a blond colour in a 300 C oven until it is very dry, and process to a fine consistency in a food processor.
	\item Break the eggs in the other platter, add a large pinch of salt, a small pinch of pepper, the finely chopped parsley, and a tablespoon of water. 
	\item Beat the eggs until they are well mixed, but not very foamy.
	\end{itemize}
	\item {\bf Cook the steaks}
	\begin{itemize}
	\item Heat up a large heavy-bottom pan (you want it fairly hot, but not smoking) -- a cast-iron pan is ideal.
	\item Add enough oil to generously coat the bottom of the pan.
	\item Add a small amount of butter.
	\item Working quickly, coat both sides of each steak in the bread crumbs, shaking off the excess.
	\item Coat both sides of the steak with the egg wash and place in the hot oil and butter.
	\item Cook for about 2 minutes in one side, turn and cook for another minute or two.
	\item Watch the intensity of the fire as you cook the steaks so that the pan does not become too hot or too cold.
	\item If you are preparing a large amount of steaks, as you finish each batch you can place on a wire rack over a rimmed baking sheet them in a 200 F oven, until you have finished them all.
	\end{itemize}
	\end{enumerate}
\end{description}
\end{document}





