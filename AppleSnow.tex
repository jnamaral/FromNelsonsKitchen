\documentclass [11pt, letterpaper] {article}
\input {headings}
\newcommand \recipeName {Apple Snow}
\chead {\recipeName}

\begin {document}
\input {title}

\begin{flushright}
{Adapted from {\it The Way to Cook} by Julia Child}
\end{flushright}


\vspace{0.3in}
\begin{description}

\item[Ingredients:]\ \\
	\begin{description}
	\item[For Apple Sauce:]\ \\
		\begin{itemize}
		\item 4 pounds (6 to 8 apples) Granny Smith, Golden Delicious, or other hard and fairly acid variety.
		\item 6 prunes
		\item 1 medium lemon
		\item 1 cinnammon stick
		\item 1/2 teaspoon ground cinnamon
		\item 1/2 teaspoon pure vanilla extract
		\item 1/4 cup of sugar (optional)
		\item 1 pinch of salt
		\end{itemize}
	\item[For Caramel:]\ \\
		\begin{itemize}
		\item 1 cup of sugar
		\item A large bowl (that can accommodate the caramel pan)  with cold water
		\item 1 cup of heavy cream
		\item 1 pinch of salt
		\item	2 teaspoon vanilla extract
		\end{itemize}
	\item[For Apple Snow:]\ \\
		\begin{itemize}
		\item 4 large egg whites (between 1/2 and 2/3 cups) at room temperature
		\item 1/2 teaspoon cream of tartar
		\item 1/4 cup of sugar (if apple sauce is unsweetened)
		\end{itemize}
	\end{description}

\item[Procedure:]\ \\
	\begin{description}
	\item[Make Apple Sauce:]\ \\
		\begin{itemize}
			\item Cut the apples in half.
			\item Add the zest of the lemon and the cinnamon stick and the prunes.
			\item Cook in a covered heavy-bottom pan over moderate heat --- stir from time to time to prevent it from sticking to the bottom and burning.
			\item Alternatively you can cook the apples in a covered dish in the oven at 300F or 325F.
			\item After the apples soften, stir and mash until the whole mixture is tender.
			\item Remove the cinnamon stick.
			\item Process the cooked apples through the middle grate of a food mill --- this will eliminate seeds, stems and large pieces of skin.
			\item Return the apple puree to the pan.
			\item Add the sugar (if using), the ground cinnamon, the pinch of salt, and the juice from the lemon.
			\item Return to the moderate heat and cook, stirring with a flat wooden spoon, until it is at the consistency that you desire.
			\item Let it cool and refrigerate or freeze.
		\end{itemize}
	\item[Make the Caramel Sauce:]\ \\
		\begin{itemize}
			\item [{\bf CAUTION}] {\bf Boiling caramel is the hottest thing that you will have in your kitchen and can cause serious burning.}
			\item	In a  frying pan, evenly sprinkle half of the sugar and cook over moderate heat. 
			\item Place the large bowl of cold water nearby.
			\item Warm up the heavy cream in the microwave or in a separate small sauce pan. It should be very hot but not necessarily boiling.
			\item Watch for hot spots where the sugar starts to melt first and sprinkle the remaining sugar over these hot spots.
			\item When the sugar starts browning in some spots, but some white sugar still remains, stir gently with a wooden spoon to ensure even melting.
			\item Keep cooking until the sugar is a deep amber colour. 
			\item When small bubbles will form in the amber sugar, immediately place the bottom of the pan on the cold water to stop the cooking.
			\item With the caramel still fairly hot and liquid, pour the hot cream over it. 
			\item Stir, if needed return to a moderate heat, until all the cream and the caramel make a homogeneous mixture.
			\item Stir in the vanilla and the salt.
			\item Pour into a small bowl and let cool. Serve at room temperature.
		\end{itemize}
	\item[Make the Apple Snow:]\ \\
		\begin{itemize}
			\item Beat the egg whites at low speed for a minute.
			\item Stir in the cream of tartar and slowly increase the speed.
			\item Beat until the egg whites form stiff shining peaks. 
			\item If using sugar, sprinkle the sugar and continue beating until sugar is incorporated.
			\item Reduce the speed to moderate and incorporate 1/2 cups of apple sauce at a time until you have used 3 cups of apple sauce.
			\item At the end the mixture should be stiff enough to hold its shape when scooped with a spoon.
			\item Spoon the apple snow on serving goblets and pour some caramel sauce on top of each serving.
		\end{itemize}
	\end{description}
\end{description}
\end{document}
