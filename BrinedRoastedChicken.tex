\documentclass[11pt,letterpaper]{article}
\input{headings}
\newcommand \recipeName {Roasted Chicken}
\chead{\recipeName}

\begin{document}
\input{title}

According to Jacques P\'epin roasting a perfect roast chicken is one of the most difficult tests for a chef. Here I adapt Jacques' own recipe from his {\it Today's Gourmet} book by adding a few steps. Brining is a signature step from America's Test Kitchen, and the removal of the wish bone is a nice step that I learned from Julia Child and that makes the carving of the chicken much easier.
 
\begin {description}

\item[Ingredients:]\ \\
\begin{itemize}
	\item 1 roasting chicken
	\item 1/4 cup of salt
	\item 1/4 cup of sugar
	\item rosemary
	\item black pepper
	\item 1 clove of garlic
	\item one pepper in vinegar (pimenta de cheiro)
	\item olive oil
\end{itemize}

\item[Procedure:]\ \\

\begin{enumerate}
\item {\bf Prepare the Chicken}
\begin{itemize}
\item Remove from package, wash it very well under running cold water, make sure to remove any blood from the cavity. 
\item Check for the kidneys. If they are still attached to the back bone inside the cavity, remove them by scraping with your finger under running water.
\end{itemize}

\item {\bf Remove the Wish Bone}
\begin{itemize}
\item Place the chicken in the counter so that the breast opening faces you. 
\item Feel for the wish bone along the front of the breast. 
\item With a paring knife, make incisions along the wish bone, and remove the wish bone. 
\end{itemize}

\item {\bf Brine the Chicken}
\begin{itemize}
\item In a deep mixing bowl mix 1/2 cup of salt and 1/4 cup of sugar with 2 quarts of water and stir well. 
\item Submerse the chicken in the brine, put in the refrigerator and let it brine for at least 2 hours, and up to 8 hours.
\end{itemize}

\item {\bf Season the Chicken}
\begin{itemize}
\item Cut some rosemary into very fine bits.
\item Cut a red pepper into tiny bits, crush with garlic and a some grindings of black pepper. 
\item Carefully loosen up the skin from the breast and thighs. You can use your fingers or a spoon, pushing the skin away from the meat, keeping back of the spoon touching the skin to avoid tearing the skin.
\item Rub the mix of herbs and spices under the skin, into the thigh joints, between the breast and the tenderloin and in the slits that you made to remove the wish bone.
\end{itemize}

\item {\bf Truss and rest the Chicken}
\begin{itemize}
\item A very simple trussing is enough. Just make sure to tie the legs and the wings against the body of the bird.
\item Let the chicken rest for the spices to penetrate in the meat for 30 min to 1 1/2 hours.
\end{itemize}

\item{\bf Preheat the oven to 425 degrees}

\item{\bf Cook on Top of the Stove}
\begin{itemize}
\item Heat a non-sticking skillet on high heat. 
\item Rub the chicken abundantly with olive oil. 
\item Place the chicken on its side in the skillet and brown it over medium to high heat for about 2 1/2 minutes. 
\item Then turn the chicken over and brown it on the other side for 2 1/2 minutes.
\end{itemize}

\item{\bf Roast in the Oven}
\begin{itemize}
\item Place the skillet, with the chicken on its side, in the 425-degree oven. 
\item Roast, uncovered, for 20 minutes. 
\item Turn the chicken onto its other side and roast it for another 20 minutes.
\item Turn the chicken onto its back, baste it with the fat that has rendered during the cooking, and roast it, breast side up, for 10 minutes.
\item The chicken is done when there is some mobility at the leg joints and the juices are running clear. The breast should register 150 F and the thighs 160 F on an instant read thermometer.
\item Remove from the oven and place it, breast side down, on a platter and let it rest for a five minutes before carving.
\end{itemize}
\end{enumerate}
\end{description}
\end{document}




