\documentclass[11pt,letterpaper]{article}
\input{headings}
\newcommand \recipeName {Sage-Parmesan Creamy Polenta}
\chead{\recipeName}

\begin{document}
\input{title}

I developed this recipe inspired on Gnocchi a la Romana. It has similar flavours to the gnocchi, but by achieving a creamy consistency and allowing the guests to spoon into their plates, it is easier to prepare because it does not require the cutting into small pieces and broiling of the gnocchi.

\begin{description}

\item[Ingredients:]\ \\
	\begin{itemize}
	\item 5 to 6 cups of whole milk
	\item 1 generous bunch of fresh sage
	\item 1 cup of cornmeal 
	\item 6 tablespoons of butter
	\item 3/4 teaspoon salt
	\item 1 cup of freshly grated real parmesan cheese
	\item Freshly Ground Pepper
	\end{itemize}

\item[Procedure:]\ \\
	\begin{enumerate}
	\item {\bf Sage-Infused Milk}
	\begin{itemize}
	\item Put 4 cups of milk in a heavy-button saucepan.
	\item Divide the bunch of sage into two halves. Put one half, whole leaves, into the milk.
	\item Add 3/4 teaspoon of salt to the milk.
	\item Bring the milk to a boil.
	\item Turn off the stove, cover the saucepan and let it stand for 10 to 15 minutes.
	\end{itemize}
	\item {\bf Cooking the Polenta}
	\begin{itemize}
	\item Uncover the pan, remove the sage from the milk and discard.
	 \item Bring the milk back to a boil.
	 \item Slowly pour the cornmeal over the boiling milk while constantly stirring with a wire whisk.
	 \item Keep stirring while the mixture thickens (about 3 minutes).
	 \item Reduce the stove to the lowest setting, cover the pan and let it simmer for 12-15 minutes. Every 3 minutes uncover the pan and stir with a wooden spoon.
	 \item Remove from the fire, the mixture should be thick, add one cup of cold milk.
	 \item  Add the grated cheese, freshly ground black pepper and two tablespoon of butter and stir well.
	 \item If not serving immediately, leave the polenta in the covered saucepan.
	 \item If the polenta has cooled significantly, close to the serving time reheat it.
	 \item If the polenta has thickened significantly add up to another cup of milk to bring it back to a creamy consistence.
	\end{itemize}
	\item {\bf Sage Brown Butter}
	\begin{itemize}
        		\item Transfer the creamy polenta to a buttered serving dish.
		\item Sprinkle the surface with freshly ground pepper.
		\item Smooth the surface with the back of a spoon. Then criss-cross it with the tip of a knife.
		\item Cut the remaining sage leaves into fine slices (make a cross-wide chiffonade with the leaves).
		\item Put the remaining four tablespoon of butter on a  frying pan (best to use a stainless steel or a light-colour non-stick saucepan because it is hard to see the colour of the brown butter on a dark pan).
		\item Cook the butter until it is a nice moderate brown colour. You will need to watch it very closely to allow it to develop the nutty flavour. When it is at the right colour, immediately remove from the fire and stir in the chiffonade of sage.
		\item Pour over the creamy polenta trying to distribute both the brown butter and the pieces of sage over the surface.
		\item Serve warm or at room temperature.
	\end{itemize}
     	\end{enumerate}         
\end{description}
\end{document}



