
\documentclass[11pt,letterpaper]{article}
\input{headings}
\newcommand \recipeName {Bahia Fish Moqueca}
\chead{\recipeName}

\begin{document}
\input{title}

Moqueca is a dish traditional of the Northeast coast of Brazil. It is
a fish stew flavored with coconut milk, palm oil, and pepper. The palm
oil is essential for the unique Moqueca flavor. The African red palm
oil that is more readly found in ethnic stores in Edmonton has a
stronger flavor than the one from Brazil. Thus, if using that oil,
reduce the amount to 2 teaspoons (one at the start and one at the
end). I prefer to use a small hot pepper from Brazil called ``Pimenta
de Cheiro'', literally ``fragant pepper''. It can be substituted by the
preserved hot peppers typically found in Chinese stores in Edmonton.

\vspace{0.3in}

\begin{description}

\item[Ingredients:]\ \\
	\begin{itemize}
	\item 2 lb of firm flesh white fish fillets (cod, halibut or sea-bass)
	\item 1 lb of shrimp with shells
	\item 1 8 oz can of clam juice
	\item 1 can of unsweetened coconut milk
	\item 1 red bell pepper
	\item 1 green bell pepper
	\item 1 large onion
	\item 1 cup of drained, seeded, and diced canned tomatoes
	\item 2 tablespoons of azeite de dend\^{e} (red palm oil) 
	\item 3 tablespoons of vegetable oil
	\item 1 small fresh red hot pepper
	\item Juice of 1 lemon
	\item Cilantro
	\item Salt and black pepper
	\end{itemize}

\item[Procedure:]\ \\
	\begin{enumerate}
	\item {\bf Marinate the Fish and the shrimp}
	\begin{itemize}
	\item Cut the fish into about 2 inch cubes.
	\item Peel the shrimp, saving the shells.
	\item Marinate the fish and shrimp, in separate bowls, (30 min. to 1 hour) with salt, black pepper, 
              lemon juice, two tablespoons of finely diced onion, and a tablespoon 
              of finely chopped cilantro.
	\end{itemize}
	\item {\bf Prepare the Shrimp Broth}
	\begin{itemize}
	\item Add one tablespoon of vegetable oil to a hot skillet 
              and saute the shrimp shells for a minute, add 1 cup of water and let it simmer for 3 minutes.
	\item Strain the shrimp shell broth.
	\item Mix the clam juice into the shrimp broth and reserve.
	\end{itemize}
	\item {\bf Prepare the Moqueca Base}
	\begin{itemize}
	\item Peel, seed the peppers and cut into strips.
	\item Cut the onions into fine rings.
	\item To a hot heavy-base saucepan, add 2 tablespoons of vegetable oil 
              and 1 tablespoon of azeite de dend\^e.
	\item Saute the bell peppers and the onions until softened.
	\item Add the hot pepper, salt, and black pepper.
	\item Add the diced tomatoes.
	\item Add the shrimp broth and clam juice mixture and let it simmer until the liquid reduces by half.
	\end{itemize}
	\item {\bf Finish the Moqueca}
	\begin{itemize}
	\item Pour the coconut milk in a microwave-safe dish and warm it in the microwave until it is very hot but not boiling.
	\item Bring the moqueca base to a fast boil and then turn of the fire and move the pot to a cool burner.
	\item With a slotted spoon, lift the fish from the marinating liquid and add to the pot.
	\item Add the peeled shrimp.
	\item Cover the pot and let it seat for two minutes (or a bit longer) until the shrimp is no longer transparent and the fish is cooked.
	\item Add the warmed-up coconut milk.
	\item Add 2 tablespoons of minced cilantro.
	\item Add 1 tablespoon of azeite de dend\^e.
	\item Transfer the moqueca to a warm serving dish.
	\item Serve with white rice.
	\end{itemize}
	\end{enumerate}
\end{description}
\end{document}



