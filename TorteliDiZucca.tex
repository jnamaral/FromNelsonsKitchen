\documentclass[11pt,letterpaper]{article}
\input{headings}
\newcommand \recipeName {Torteli Di Zucca}
\chead{\recipeName}

\begin{document}
\input{title}

\begin {flushright}
Adapted from Lidia Bastianich.
\end {flushright}

Then we go to the Reggio Emilia region of Italy, which is located to the East of Parma and to the West of Modena. From there we will taste {\it Tortelli di zucca alla Mantovana}, which is a delicate tortelli filed with a roasted pumpkin puree that is flavoured with Mostarda di Cremona, a sweet and spicy fruit in sirup. Cremona is not far to the North of the Reggio Emilia region. Note:
In Edmonton you find the Mantovana squash at Superstore, at T\&T, or at Lucky 97. I use butter and milk in the pasta to make a tender dough that is more suitable to the delicate squash filling. The Mostarda di Cremona has to be mail order or acquired at an Italian store elsewhere. It is important to prepare both the pasta and the filling several hours before forming the raviolis.

\begin{description}

\item[Ingredients:]\ \\
	\begin{itemize}
	\item 6 eggs
	\item 3 Tbs melted unsalted butter
	\item 3 Tbs whole milk
	\item 1/2 teaspoon salt
	\item 4 cups unbleached all-purpose flour
	\item	2 1/2 pounds of Mantovana squash
	\item 1 cup of freshly grated Parmigiano-Reggiano cheese
	\item 1/2 cup finely diced mostarda di Cremona
	\item 8 amaretti cookies, crushed to fine crumbs (about 2/3 cup)
	\item Freshly grated nutmeg
	\item Freshly ground black pepper
	\item 2 large egg yolks
	\item 5 to 7 Tbs unsalted butter
	\item Fresh sage leaves
	\item 2/3 to 1 cup grated Parmigiano-Reggiano cheese
	\item 1 Tbs of salt
	\end{itemize}

\item[Procedure:]\ \\

	\begin{enumerate}
	\item {\bf Prepare the pasta}
	\begin{itemize}
	\item Break the eggs into a large bowl and mix with a fork to break the yolks.
	\item While mixing the eggs pour the melted butter on a slow stream.
	\item Add the milk and the salt and mix.
	\item Slowly start incorporating the flour until you can no longer mix with the fork.
	\item Add a bit more flour and start incorporating it with your hand.
	\item Turn dough into a flat surface and keep kneading and incorporating flour until the dough is slightly firm and fairly smooth.
	\item Spray the bowl with cooking spray, put the dough ball into the bowl, and cover with plastic wrap
	\item Let rest for at least one hour at room temperature.
	\item Put dough on flat surface and knead again until it is very smooth.
	\item Put dough ball back in sprayed bowl and cover with plastic wrap to rest for at least one hour or until you are ready to form the raviolli.
	\end{itemize}

	\item {\bf Make the Squash Puree}
	\begin{itemize}
	\item Preheat oven to 375 F.
	\item Wash the squash.
	\item Cut into quarters and then cut each quarter in half crosswise.
	\item Scrape out the seeds.
	\item Arrange squash pieces on a baking sheet cut side up.
	\item Bake until  it feels soft and is easily pierced with a knife, between 40 min. and one hour. 
	\item Let the Squash cool.
	\item Scoop the flesh into the bowl of food processor. 
	\item Process until it is smooth.
	\item Transfer to a mixing bowl and refrigerate for at least 30 minutes.
	\end{itemize}

	\item {\bf Mix the Filling}
	\begin{itemize}
	\item Add the diced mostarda, grated cheese and crushed amaretti.
	\item Season with nutmeg, pepper, and salt to taste.
	\item Beat in the egg yolks until smooth and well blended.
	\end{itemize}

	\item {\bf Form the Raviolli}
	\begin{itemize}
	\item Line baking sheet with parchment paper.
	\item Cut 1/3 of the dough and keep the remainder dough covered with the plastic wrap.
	\item Roll out 1/3 of the dough with a rolling pin forming a rectangular shape until the dough is fairly thin. Use an abundant amount of flour when rolling.
	\item Brush the top side of the dough with a pastry brush to remove all the excess flour. 
	\item Drop about half a tablespoon of filling along one side of the dough in regular intervals, leaving enough of a margin to allow the end of the dough to fold over the mounds of filling.
	\item Using your finger and a small dish with cold water, paint water around each mound of filling.
	\item Dry your hand, fold the margin of dough over the mounds of filling and press firmly around each feeling.
	\item Using a pastry cutting wheel, cut the squares of raviolli around each mound of filling.
	\item Place form raviolli on parchment-lined baking sheets well spaced from each other.
	\item Repeat the process until all the rolled-out dough is used.
	\item Repeat with the remainder the dough, rolling out 1/3 each time.
	\item If not cooking soon, after a while turn raviolli over in the parchment-lined baking sheets so that they do not stick to the paper.
	\end{itemize}

	\item {\bf Cook the Raviolli and Make the Sauce}
	\begin{itemize}
	\item Bring a large pot of water to boil. Add one tablespoon of salt.
	\item Turn oven to 200 F and put serving plates in the oven.
	\item Cut sage leaves into fine strips --- it is called a chiffonade.
	\item Drop the raviolli into the boiling water.
	\item Put butter in a large non-sticking pan and put over medium heat.
	\item When the butter has melted and is start foaming, drop the chiffonade of sage into the butter and stir. It should crisp up in less then a minute.
	\item When the sage is crisp and the butter has turned to a pale brown colour, if the raviolli is not ready yet, add a few tablespoons of the cooking water to the butter to slow the cooking and lower the heat.
	\item The raviolli should be cooked about a minute after they float to the surface. The total cooking time will be between 4 and 5 minutes.
	\item Using a mesh scoop the raviolli from the boiling water straight into the pan with butter and sage.
	\item Sautee for a few minutes shaking the pan often.
	\item Add enough of the cooking water to make a small amount of a thick sauce.
	\item Serve immediately into warmed up plates.
	\item Sprinkle the freshly grated Parmigianno-Regiano over each plate.
	\item Sprinkle a small amount of freshly grated pepper.
	\end{itemize}

	\end{enumerate}
\end{description}
\end{document}



