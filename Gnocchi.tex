\documentclass[11pt,letterpaper]{article}
\input{headings}
\newcommand \recipeName {Gnocchi}
\chead{\recipeName}

\begin{document}
\input{title}

\begin{flushright}
{\bf From {\it Lidia's Italian Table} by Lidia M. Batianich.}
\end{flushright}

\begin{description}

\item[Ingredients:]\ \\
	\begin{itemize}
	\item 1 3/4 pounds of potatoes 
	\item 1 large egg
	\item 1 teaspoon of salt 
	\item 1/4 teaspoon of ground white pepper 
	\item Pinch of freshly grated nutmeg
	\item 1/4 cup of freshly grated parmigiano-Reggiano cheese 
	\item 2 cups unbleached all-purpose flour, or as needed
	\item 2 tablespoons of chopped Italian parsley 
	\item 4 tablespoons of butter (optional for sauce)
	\end{itemize}

\item[Procedure:]\ \\
	\begin{enumerate}
	\item {\bf Cook the potatoes}
	\begin{itemize}
	\item Place the potatoes in a large pot with enough cold water to cover.
        \item Bring the water to a boil and cook, partially covered, until the potatoes are easily pierced with a skewer but the skins are not split (35 minutes).
	\end{itemize}
	\item {\bf Rice the potatoes}
	\begin{itemize}
	\item Drain the potatoes and let them stand until cold enough to handle.
	\item Holding the potatoes with a kitchen towel or mitt, peel the potatoes with a paring knife.
	\item Pass the potatoes through a food mill fitted with the fine disc letting them fall on top of the counter.
	\item Spread the riced potatoes into a thin layer on the work surface, without pressing them or compacting them. Let them cool completely
	\end{itemize}
	\item {\bf Make the egg mixture}
	\begin{itemize}
	\item In a small bowl, beat the egg, salt, pepper and nutmeg together.
	\end{itemize}
	\item {\bf Prepare the dough}
	\begin{itemize}
	\item Spread the grated Parmesan cheese over the cold spreaded potatoes.
	\item Gather the potatoes and cheese on a mound with a well in the center.
	\item Pour the egg mixture in the well.
	\item Knead the potato and egg mixture adding enough flour to make a smooth, but slightly sticky dough. Avoid adding too much flour to make a light gnocchi.
	\item Use a dough scrapper to scrape the dough from your hands and working surface as you knead.
	\end{itemize}
	\item {\bf Form the gnocchi}
	\begin{itemize}
	\item Form a rectangular shape about 3/4 inch thick on the counter with the dough.
	\item Cut the dough into 3/4 inch squares.
	\item Form a rough ball with each square by lightly rolling each one in your hands.
	\item Hold the tines of a fork at a 45-degree angle to the table with the concave part facing up.
	\item Take each ball of dough and with the tip of your thumb, press the dough lightly against the tines of the fork as you roll it downward towards the tip of your thumb. As the dough wraps around the tip of your thumb, it will form into a dumpling with a deep indentation on one side and a ridged surface on the other.
	\item Set the gnocchi on a baking sheet lined with a floured kitchen towel.
	\end{itemize}
	\item {\bf Pre-Cook the gnocchi}
	\begin{itemize}
	\item Oil or butter a baking sheet to contain the pre-cooked gnocchi.
	\item Bring 6 quarts of salted water to a vigorous boil over high heat.
	\item Drop small batches of gnocchi into the boiling water.
	\item Cook, stirring gently with a wooden spoon, until tender (about a minute).
	\item Remove the gnocchi from the water with a slotted spoon, drain them well and transfer them to the oiled baking sheet.
	\end{itemize}
	\item {\bf Finish the gnocchi (with butter)}
	\begin{itemize}
	\item When ready to serve, add butter to a saute pan, heat it up until the butter acquires a golden color. Be careful to not burn the butter.
	\item Add the pre-cooked gnocchi to the brown butter and saute until they start to acquire some gold color.
	\item Sprinkle with chopped parsley.
	\item Sprinkle with grated Parmesan cheese.
	\item Serve immediately.
	\end{itemize}
	\item {\bf Finish the gnocchi (with roasted chicken juices)}
	\begin{itemize}
	\item When ready to serve, remove the chicken from the skillet in which it was roasted.
	\item Skim some of the excess fat from the skillet, leaving the juices in it.
	\item Bring the juices and remaining fat to a vigorous boil in the sauce pan.
	\item Add the pre-cooked gnocchi to the brown butter and saute until they start to acquire some gold color.
	\item Sprinkle with chopped parsley.
	\item Sprinkle with grated Parmesan cheese.
	\item Serve immediately.
	\end{itemize}
	\end{enumerate}

\end{description}
\end{document}




