\documentclass[11pt,letterpaper]{article}
\input{headings}
\newcommand \recipeName {BBQed Chicken Drumsticks}
\chead{\recipeName}

\begin{document}
\input{title}

The origins of this recipe was a Windsor-Park Community League BBQ in 2017. My husband suggested that instead of traditional frozen hamburger patties, the community should try something different. We offered to contribute some BBQed drumsticks and I created this recipe for the event. It was a success and many people asked me for the recipe.

\begin{description}

\item[Ingredients:]\ \\
	\begin{itemize}
	\item 12 chicken drumsticks
	\item 1/4 cup of table salt
	\item 1/4 cup of sugar
	\item 2 Tbs chili powder
	\item 1 Tbs paprika
	\item 1 Tbs cumin seeds
	\item 1 teaspoon whole black peppercorn kernels
	\item Cooking spray
	\item 1 cup of barbecue sause (I use Bull's Eye)
	\item 1/2 cup of ketchup
	\end{itemize}
\item[Equipment and Supplies:]\ \\
	\begin{itemize}
	\item two rimmed baking sheets 
	\item a rack that fits in one of the baking sheets
	\item a small coffee grinder that you use for spice grinding
	\item aluminum foil
	\end{itemize}

\item[Procedure:]\ \\
	\begin{enumerate}
	\item {\bf Prep chicken and brine it:}
		\begin{itemize}
		\item Using a sharp paring knife cut all the skin and tendons at the base of the bone of each drumstick. This is an important step because it will allow the meat to contract as it cook making for a more tender and meaty drumstick. 
		\item Wash the drumstick in cold water and discard the water.
		\item In  a container that will go into the fridge,  mix 2 quarts of water with 1/4 cup of table salt and 1/4 cup of sugar and submerge the drumsticks in the water.
		\item Refrigerate for at least one hour, but you can leave the chicken in the brining solution overnight.
		\end{itemize}
	\item {\bf Air dry the chicken:}
		\begin{itemize}
		\item Place a rack in a rimmed baking sheet.
		\item Remove drumsticks from brining solution and place on the rack. Stretch the skin of each drumstick over the leg meat.
		\item Put in the refrigerator, uncovered, for at least a couple of hours, but you can leave in the fridge for up to 12 hours. 
		\end{itemize}
	\item {\bf Season the Chicken with a Dry Rub}
		\begin{itemize}
		\item Put a small dry skillet on the stove over moderate heat.
		\item Put the cumin seeds and the whole black peppercorn kernels in the skillet.
		\item Dry roast these spices for a few minutes until they are very aromatic but not burned.
		\item Dump the roasted spices in a small coffee grinder that you use for spice grinding.
		\item Grind the freshly roasted spices to a powder and dump into a small bowl.
		\item Add the chilli powder and the paprika to the spice bowl and stir well.
		\item Remove the air-dried chicken from the rimmed baking sheet and put in a container with cover.
		\item Sprinkle the spice mixture over the drumsticks and stir around so that they are all well coated.
		\item Cover and refrigerate for at least one hour, but you can refrigerate overnight.
		\end{itemize}
	\item {\bf Slow roasting the chicken}
		\begin{itemize}
		\item Coat a rimmed baking sheet with cooking spray.
		\item Place the drumsticks into the baking sheet. Again if skins have retracted, stretch the skin over the meat of each drumstick.
		\item Heat oven to 325F and roast the chicken for 1 1/2 to 2 hours. The meat will have pulled away from the bone at the bottom. The chicken should be cooked, but still juicy. It will not have browned much.
		\item Chicken will release juices during this baking. Do not throw away the juice.
		\end{itemize}
	\item {\bf Dry drumstick's skin in high oven}
		\begin{itemize}
		 \item Remove the chicken from the oven.
		 \item Increase the oven temperature to 425 F.
		 \item Cover a rimmed baking sheet with aluminium foil.
		 \item Over the sink, spray a rack with cooking spray.
		 \item Place the rack over the aluminium-covered baking sheet.
		 \item Place the drum sticks on the rack making sure that the side that was the bottom side from the earlier baking is now facing up.
		 \item Roast for 15-20 minutes until the skin is dry. The chicken may still not have browned much.
		 \end{itemize}
	\item {\bf Coat with barbecue sauce}
		\begin{itemize}
		\item Mix barbecue sauce with ketchup and about 1/2 cup of the liquid that was released from the first baking of the chicken in a small bowl.
		\item Carefully dip each drumstick into the barbecue sauce mixture ensuring to coat only the meat but not the exposed bone and the knob at the end of the bone.
		\item Place the sauce-coated drumsticks back onto the rack on top of the aluminum-covered baking sheet.
		\item Return to the oven for another 10 minutes or so until the sauce has dried a bit but is still shiny.
		\item Let cool for several minutes before serving. 
		\end{itemize}
     	\end{enumerate}         
\end{description}
\end{document}



