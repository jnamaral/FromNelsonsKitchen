\documentclass[11pt,letterpaper]{article}
\input{headings}
\newcommand \recipeName {GrilledPorkTenderloinSpiceMarinade}
\newcommand \fileName {GrilledPorkTenderloinSpiceMarinade}

\chead{\recipeName}

\begin{document}
\input{title}

%\begin{flushright}
%{\bf From {\it } by }
%\end{flushright}
 
 <p>This brined and grilled pork tenderloin is a staple at my house. I make some variations. This recipe uses the Spice Marinade that appears in {\it The Way to Cook} by Julia Child. This  recipe makes more spice marinade than you need for this recipe. You can keep it for later use.</p>
 
 <p>Two simple variations consist in (1) use chopped up rosemary mixed with cracked black pepper, you can crack the pepper on the kitchen counter under the bottom of a clean heavy pan; (2) use Montreal steak spice mixture.
 
\begin{description}

\item[Ingredients (Spice Marinade):]\ \\
	\begin{itemize}
	\item 2 tablespoon of each of the following ground spices:
		\begin{itemize}
		\item bay leaf
		\item clove
		\item mace
		\item nutmeg
		\item paprika
		\item thyme
		\end{itemize}
	\item 1 tablespoon of each of the following ground spices:
		\begin{itemize}
		\item allspice
		\item cinnamon
		\item savory
		\end{itemize}
	\item 5 tablespoon of ground white peppercorns
	\end{itemize}

\item[Ingredients:]\ \\
	\begin{itemize}
	\item pork tenderloins
	\item salt
	\item sugar
	\item cooking spray
	\end{itemize}

\item[Procedure:]\ \\
	\begin{enumerate}
	\item {\bf Trim, brine, and air dry the pork tenderloins}
	\begin{itemize}
	\item Remove the silver skin from the tenderloins using a boning knife
        \item Cut off the thin ends of the tenderloins.
	\item Make a brining solution with the following proportions: for each  two quarters of water, add 1/4 cup of table salt and 1/4 cup of sugar.
        \item Brine the tenderloins for at least two hours, but you can brine it overnight.
        \item Place the tenderloins on a baking tray and put in the refrigerator for several hours to dry the outside surfaces.
	\end{itemize}
	\item {\bf Marinate the tenderloins}
	\begin{itemize}
	\item Lightly spray the tenderloins with cooking spray.
	\item Spread some of the marinade on a plate and roll the tenderloins on the marinade until each tenderloin is fully coated with the marinade.
	\end{itemize}
	\item {\bf Grill the tenderloins}
	\begin{itemize}
	\item Get your grill as hot as you can (my gas grill gets to 550F)
	 \item Brush the grill with a hard grill brush to make sure it is clean
	 \item Pour a small amount of oil in a small dish, fold a piece of paper towel several times and holding the folded paper towel with kitchen tongues, deep it in the oil and smear all over the grill. Cover the grill to let the oil burn for a minute. Repeat the process three or four times to reduce the stickiness of the grill. 
	\item Reduce the temperature of the grill to a moderate high heat.
	\item Place the tenderloin on the grill.
	\item Keep turning as it browns in each side.
	\item Grill until it register 140F on an instant thermometer.
	\end{itemize}
	\item {\bf Cool and serve}
	\begin{itemize}
	\item As soon as you remove the tenderloins from the grill, put on a covered container and let rest for ten minutes.
	\item Slice  on 1/2 inch slices and serve.
	\end{itemize}
	\end{enumerate}
\end{description}
\end{document}



