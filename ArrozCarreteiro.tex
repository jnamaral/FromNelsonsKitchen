\documentclass[11pt,letterpaper]{article}
\input{headings}
\newcommand \recipeName {Arroz de Carreteiro}
\chead{\recipeName}

\begin{document}
\input{title}

Arroz de Carreteiro is a traditional dish from the state of Rio Grande do Sul. There are probably as many recipes for Arroz Carreteiro as there are Gaucho cooks. Here is one that I use. I make my own ``charque" because I live in Edmonton, Alberta, but in Rio Grande do Sul you can buy the charque already made.

\vspace{0.3in}

\begin{description}

\item[Ingredients:]\ \\
	\begin{itemize}
	\item	2 pounds of sirloin 
	\item Coarse salt
	\item Kitchen twine
	\item 3 Tbs of Cooking oil
	\item 1 cup fo chopped onions
	\item 2 cups of peeled can tomatoes (chopped)
	\item 3 cups of rice
	\item 6 cups of hot water
	\end{itemize}

\item[Procedure:]\ \\

	\begin{enumerate}
	\item {\bf Make the Charque (a few days before you plan to make the Carreteiro)}
	\begin{itemize}
	\item Trim fat and sinuous tissue from the meat.
	\item Gently rub coarse salt on the meat until it is covered in salt.
	\item Make a small slit close to one end of each piece of meat with a paring knife.
	\item Thread a long piece of twine and tie.
	\item Tie the twine to something high and put a large container underneath the pieces of meat.
	\item Let the slated meat hang for a few days. Check every day to ensure that the meat is not getting too dry (it should not become leathery).
	\item When the meat has dried enough and acquired a dark purple colour, you can place the meat into plastic bags and store in the refrigerator for up to a few weeks, or in the freezer for months.
	\end{itemize}
	\item {\bf Dice the charque}
	\begin{itemize}
	\item Rinse the charque under cold water and remove any visible salt that might be stuck to the charque.
	\item Dice the charque in very small pieces (about 1/4 of an inch)
	\end{itemize}
	\item {\bf Sautee and boil the rice}
	\begin{itemize}
	\item Heat the water in the microwave until it is boiling --- four to five minutes.
	\item In a large heavy pot, heat up some cooking oil.
	\item Sautee the charque, but don't let it get too crispy.
	\item Add the chopped onions and sautee until they are soft.
	\item Add the chopped tomatoes and cook until most of the liquid has evaporated.
	\item Add the rice and stir fry it until some of the grains start becoming opaque.
	\item Add the hot water.
	\item Taste the water to see if you need to add more salt.
	\item Reduce the fire to the slowest simmer.
	\item Pour the hot water over the rice stirring.
	\item Cover the pot and simmer for about 35 minutes. Do not open the pot.
	\item After 35 minutes, taste a few grains from the top to see if they are almost cooked. If they are, turn off the fire and keep the pot closed for another ten minutes.
	\item Do not stir the rice.
	\end{itemize}
	
	\end{enumerate}
\end{description}
\end{document}



