\documentclass[11pt,letterpaper]{article}
\input{headings}
\newcommand \recipeName {Hard Boiled Eggs}
\chead{\recipeName}

\begin{document}
\input{title}

\begin{flushright}
{\bf From {\it The Way to Cook } by Julia Child }
\end{flushright}
 
 Boiling an egg is considered one of the most trivial tasks in a kitchen. Often we hear the joke ``at least she knows how to boil an egg" or ``he cannot even boil an egg." Well truth is that, if you care for the result as you should, boiling a hard-boil egg is a simple but remarkable tricky cooking task. And once you start paying attention like I do, you will see many restaurants, some in expensive hotels, that failed at this task. A perfectly boiled hard egg shall have not even the faintest darkening of the yolk skin. The darkening of the skin is due to the formation of sulfur when the yolk stays for too long at a high temperature. This procedure here was developed by Georgia Egg Board and appears in Julia Child's {\it The Way to Cook}.  
 
\begin{description}

\item[Ingredients:]\ \\
	\begin{itemize}
	\item Up to four eggs
	\item 2 quarts of water
	\item If cooking 5 to 12 eggs use 3 1/2 quarts of water
	\end{itemize}

\item[Procedure:]\ \\
	\begin{enumerate}
	\item {\bf Cook the eggs }
	\begin{itemize}
	\item You will need a bowl with icy cold water with ice by the end of this cooking time.
	\item If you will serve the eggs whole or in halves and want them to have a nice shape, make a tiny hole with a needle in the larger end of the egg for the air to escape. Do not break the internal membrane
	\item Gently lay the eggs into the cold pan.
        \item Cover the eggs with cold water.
	\item Set over high heat and bring just to a boil.
	\item Remove from heat.
	\item Cover the pan, and let sit for exactly 17 minutes.
	\end{itemize}
	\item {\bf Two-minute Chill}
	\begin{itemize}
	\item Transfer the eggs to the bowl of icy cold water.
	\item Immediately put the pot where the eggs were boiled back under high heat.
	\item Let the eggs chill in the cold water for exactly two minutes.
	\end{itemize}
	\item {\bf Ten-seconds boil}
	\begin{itemize}
	\item Transfer the eggs, a few at a time, to the now boiling water.
	\item Let the eggs be in the boiling water for exactly 10 seconds --- the reason is to expand the shells and make it easier to peel.
	\item Crack the shells gently in several places and submerge the eggs into the icy cold water.
	\item Leave the eggs in the icy water for 15 to 20 minutes before peeling 
	\end{itemize}
	\item {\bf Peel the eggs}
	\begin{itemize}
	\item Crack the egg shell all over by gently tapping it against the counter.
	\item Hold the egg under a thin stream of water.  
	\item Starting peeling on the large end.
	\item Return each egg back into the icy water as soon as it is peeled.
	\end{itemize}
	\item {\bf Storing}
	\begin{itemize}
	\item To store, keep the eggs submerged in cold water in the refrigerator and keep it uncovered.
	\end{itemize}
	\end{enumerate}
\end{description}
\end{document}



