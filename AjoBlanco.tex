\documentclass[11pt,letterpaper]{article}
\input{headings}
\newcommand \recipeName {Ajoblanco}
\chead{\recipeName}

\begin{document}
\input{title}

We start in Malaga in the very South of Spain, just Northeast to Gibraltar with a very traditional Ajoblanco de Malaga, which is also known as a Spanish White Gazpacho. It is a creamy cold white soup. The most prominent flavours in Ajoblanco are almonds, olive oil, and garlic. The traditional recipe features raw garlic which would give a prominent assertive garlic flavour. I cannot tolerate that flavour. Thus my adaptation is to roast the garlic gently in olive oil to achieve a much milder and sweeter garlic flavour --- citizens of Malaga would likely protest this change, but this is my Brazilian heritage in action.


\begin{description}

\item[Ingredients:]\ \\
	\begin{itemize}
	\item	6 slices hearty white sandwich bread, crusts removed
	\item 4 cups water
	\item 2 1/2 cups (8 3/4 ounces) plus 1/3 cup sliced blanched almonds
	\item 2 garlic cloves, peeled
	\item 3 tablespoons of sherry vinegar
	\item Kosher salt and pepper
	\item Pinch cayenne pepper
	\item 1/2 cup extra-virgin olive oil, plus extra for drizzling
	\item 1/8 teaspoon almond extract
	\item 2 teaspoons vegetable oil
	\item 6 ounces seedless green grapes, sliced thin (1 cup)
	\end{itemize}

\item[Procedure:]\ \\

	\begin{enumerate}
	\item {\bf Roast the Garlic}
	\begin{itemize}
	\item Put two tablespoons of the olive oil into your smallest saucepan.
	\item Slice the cloves of garlic on half and add to the saucepan.
	\item Put over a slow simmer and roast until garlic turns a pale blond colour. 
	\item Immediately pour the garlic and garlic cloves into a cold small dish and reserve.
	\end{itemize}
	
	\item {\bf Soak the Bread}
	\begin{itemize}
	\item Combine bread and water in bowl and let soak for 5 minutes. 
	\end{itemize}
	
	\item {\bf Process Almonds and Bread}
	\begin{itemize}
	\item Process 2 1/2 cups almonds in food processor until finely ground, about 30 seconds, scraping down sides of processor as needed.
	\item Using your hands, remove bread from water, squeeze it lightly, and transfer to food processor with almonds. Reserve the water.
	\item Add roasted garlic with its oil, vinegar, 1 1/4 teaspoons salt, and cayenne to blender and process until mixture has consistency of cake batter, 30 to 45 seconds. 
	\end{itemize}	
	
	\item {\bf Add Olive Oil}
	\begin{itemize}
	\item  With processor running, add olive oil in thin, steady stream, about 30 seconds. 
	\item Add reserved soaking water and process for 1 minute. 
	\item Season with salt and pepper to taste. 
	\end{itemize}
	
	\item {\bf Strain the Soup}
	\begin{itemize}
	\item Strain soup through fine-mesh strainer set in bowl, pressing on solids to extract liquid.
	\item Discard solids.
	\end{itemize}
	
	\item {\bf Adding Almond Extract}
	\begin{itemize}
	\item Measure 1 tablespoon of soup into second bowl and stir in almond extract. 
	\item Return 1 teaspoon of extract mixture to soup; discard remainder. 
	\item Chill for at least 3 hours or up to 24 hours.
	\end{itemize}
	
	\item {\bf Toast Almonds}
	\begin{itemize}
	\item Warm oven to 300 F.
	\item Mix remaining 1/3 cup of sliced almonds with the vegetable oil and spread on a light-coloured baking sheet.
	\item Toast almonds, stirring from time to time, until they are lightly toasted.
	\item Remove and let it cool off.
	\end{itemize}
	
	\item {\bf Serve the Soup}
	\begin{itemize}
	\item  Ladle soup into shallow bowls. 
	\item Mound an equal amount of grapes in center of each bowl. 
	\item Sprinkle cooled almonds over soup and drizzle with extra olive oil. Serve immediately.
	\end{itemize}

	\end{enumerate}
\end{description}
\end{document}



