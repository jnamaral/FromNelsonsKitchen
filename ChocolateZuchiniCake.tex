\documentclass[11pt,letterpaper]{article}
\input{headings}
\newcommand \recipeName {Chocolate Zuchini Cake}
\chead{\recipeName}

\begin{document}
\input{title}


\begin{flushright}
{\bf Adapted from {\it Bon App\'etit}, November 1995. }
\end{flushright}

The adaptations here are to make  the recipe dairy free. I replaced buttermilk for a combination of almond milk and lemon juice, and the butter for vegetable oil.

\begin{description}

\item[Ingredients:]\ \\
	\begin{itemize}
	\item 2 1/4 cups all purpose �our (11 oz 3/4, 330 grams)
	\item 1/2 cup unsweetened cocoa powder  (1 1/2 oz, 45 grams)
	\item 1 teaspoon baking soda   
	\item 1/2 teaspoon of baking powder
	\item 1 3/4 cups sugar                  (11 3/4 oz, 335 grams)
	\item 1  cup vegetable oil             (8 oz, 230 grams)
	\item 2 large eggs 
	\item 1 teaspoon vanilla extract 
	\item 1 teaspoon of lemon juice
	\item 1 teaspoon salt 	         
	\item 2 cups grated unpeeled zucchini (about 2 1/2 medium)  (10 oz, 300 grams)
	\item 1/2 cup almond milk
	\item 1 6-ounce package (about 1 cup) bittersweet chocolate chips or chopped unsweetened chocolate bars (70\% cocoa)
	\item 3/4 cup chopped walnuts (optional)
	\end{itemize}
	
\item[Procedure:]\ \\

	\begin{enumerate}
	\item {\bf Preheat oven and prepare the pan}
	\begin{itemize}
	\item Preheat oven to 325 F.
	\item Cut a rectangular piece of parchment paper that covers the bottom of a $ \times 9 \times 2$ rectangular pan and that comes up the long side of the pan.
	\item Spray the bottom of the pan with cooking spray.
	\item Line the pan with the parchment paper and spray the parchment paper with the cooking spray.
	\end{itemize}
	\item {\bf Grate the zuchini}
	\begin{itemize}
	\item Grate the zuchini in the large holes of a box grater or in a food processor equipped with the disc with large holes.
	\end{itemize}
	\item {\bf Sift the dry ingredients}
	\begin{itemize}
	\item Cut a large square of parchment paper and lay on top of counter
	\item Using a large strainer, sift the flour, cocoa power, baking soda and baking powder on top of the parchment paper.
	\end{itemize}
	
	\item {\bf Mix the batter}
	\begin{itemize}
	\item In a large bowl, using a whisk, mix the sugar, oil, eggs, vanilla, lemon juice, and salt.
	\item Add 1/3 of the dry mixture and stir gently until incorporated followed by 1/3 of the almond milk. Repeat until all the dry ingredients and milk have been incorporated.
	\item Add the grated zuchini and mix until incorporated.
	\item Add the chocolate chips, or chopped chocolate, and nuts (if using).
	\end{itemize}

	\item {\bf Bake, cool and serve}
	\begin{itemize}
	\item Pour the batter in the prepared loaf pan.
	\item Bake at 325 F for 50 minutes to one hour or until a toothpick comes out clean when inserted in the middle of the loaf. Rotate the pan in the oven after 25 minutes to ensure even baking.
	\item Remove from oven and let it cool, in the pan,  on a rack for 10 minutes.
	\item Run a sharp knife along the short ends of the pan that were not covered by the parchment paper to release the bread.
	\item Invert on top of a cutting board.
	\item Peel the parchment paper from the bread.
	\item Invert again (you want the top side up now) on top of the cooling rack. 
	\item Serve warm or let cool to room temperature.
	\item If it is completely cool when you serve, you may warm up a slice for 20 seconds in full power in the microwave. Best is to have a plastic cover over the slice. 
	\item Serve warm with a spoonful of sour cream, creme fraiche,  or with some whipped cream.
	\end{itemize}
	
	\end{enumerate}
\end{description}
\end{document}



