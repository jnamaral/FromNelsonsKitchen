\documentclass[11pt,letterpaper]{article}
\input{headings}
\newcommand \recipeName {Harissa}
\chead{\recipeName}

\begin{document}
\input{title}

Harissa is a spicy paste that is used as a seasoning in other recipes. It can be found in specialized food stores. But the freshly made one with fresh garlic and fresh-toasted whole spices is much superior. This recipe is from \href{https://www.thekitchn.com/how-to-make-harissa-cooking-lessons-from-the-kitchn-190188}{www.thekitchn.com}.
\begin{description}

\item[Ingredients:]\ \\
	\begin{itemize}
	\item 4 ounces dried chiles of your choice
	\item 1 teaspoon caraway seeds
	\item 1 teaspoon coriander seeds
	\item 1 teaspoon cumin seeds
	\item 3 to 4 cloves garlic, peeled
	\item 1 teaspoon kosher salt, or to taste
	\item 2 tablespoons extra virgin olive oil, plus more for storing
	\item 2 tablespoon of fresh lemon juice
	\end{itemize}

\item[Equipment:]\ \\
	\begin{itemize}
	\item Skillet 
	\item Spice grinder or coffee grinder
	\item Small food processor
	\end{itemize}

\item[Procedure:]\ \\
	\begin{enumerate}
	\item {\bf Soften the chiles}
	\begin{itemize}
	\item Place the chiles in a heatproof bowl and barely cover with boiling water. Let stand for 30 minutes.
	\end{itemize}
	\item {\bf Toast and grind the spices}
	\begin{itemize}
	\item Toast the caraway, coriander, and cumin in a dry skillet over low-medium heat, shaking or stirring to prevent burning. 
	\item When the spices are fragrant, immediately remove them from the pan into a coffee grinder.
	\item Grind the spices in the grinder to obtain a fine powder.
	\end{itemize}
	\item {\bf Prepare the Chiles}
	\begin{itemize}
	\item Drain the chiles, reserving the liquid.
	\item Remove and discard the stems and seeds from the chiles. Wear latex gloves to protect your hands.
	\end{itemize}
	\item {\bf Make the paste}
	\begin{itemize}
	\item Put the chiles into a small food processor.
	\item Peel the garlic and roughly crush with a chef's knife over a wooden board.
	\item Add the ground spices.
	\item Add the salt.
	\item Process to obtain a smooth and thick paste.
	\item While the processor is running slowly drizzle in the olive oil.
	\item  If you want a thinner paste, blend in a little of the chile-soaking liquid until the paste has reached your desired texture.
	\end{itemize}
	\item {\bf Store}
	\begin{itemize}
	\item Rinse a clean jar, and the cover, with boiling water.
	\item Transfer the harissa paste to the jar.
	\item Smooth the surface and cover the surface with a thin layer of olive oil.
	\item Cover the jar and refrigerate. It will keep for up to a month. Add a thin layer of olive oil whenever you use the harissa.
	\end{itemize}
	\end{enumerate}
\end{description}
\end{document}
