\documentclass[11pt,letterpaper]{article}
\input{headings}
\newcommand \recipeName {Daniel's Belgian Waffles}
\chead{\recipeName}

\begin{document}
\input{title}


This is a dairy-free version of Belgium waffles that was adapted from the {\it Better Homes and Gardens Cookbook}. Prepare the batter the night before and cook the waffles in the morning. The batter stays in the fridge for a day. If keeping for a longer time, add a teaspoon of sugar and a pinch of baking soda right before cooking --- the yeast eats up all the sugar and turns the batter slightly sour, the baking soda compensates for that and helps with browning.

\begin{description}

\item[Ingredients:]\ \\
	\begin{itemize}
	\item 1 3/4 cups (14 oz) almond milk
	\item 2 Tbs (3/4 oz) sugar
	\item 1 Tbs yeast
	\item 1/3 cup (66 grams) unflavoured oil
	\item 2 eggs
	\item 1/2 teaspoon salt
	\item 2 1/4 cup (11 1/4 oz) flour
	\item 1/4 teaspoon cinnamon
	\end{itemize}

\ \\
\item[Equipment:]\ \\
\begin{itemize}
\item One whisky balloon.
\item One spatula.
\item One waffle iron.
\end{itemize}
\ \\


\item[Procedure:]\ \\

	\begin{enumerate}
	\item {\bf Mix the batter}
	\begin{itemize}
	\item In a bowl or large measuring container with a pouring spout measure the almond milk, add the sugar and the yeast, mix briefly and let it stand for 5 to 10 minutes to activate the yeast.
	\item Add the oil, eggs, salt and mix well with a whisk.
	\item Add the flour and the cinnamon and mix well with the whisk but do not over mix. The batter should be lumpy.
	\item Cover the recipient with plastic wrap and put in the refrigerator overnight.
	\end{itemize}

	\item {\bf Cook the waffles}
	\begin{itemize}
	\item Pour batter in an electric waffle iron set to the cooking setting desired.
	\item Spread a small amount of margarine (or butter) on top of each waffle and serve immediately with maple sirup. 
	\end{itemize}

	\end{enumerate}
\end{description}
\end{document}



