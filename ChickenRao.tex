\documentclass[11pt,letterpaper]{article}
\input{headings}
\newcommand \recipeName {Chicken Rao}
\chead{\recipeName}

\begin{document}
\input{title}


Signature dish from New York's Rao restaurant. Adapted from {\it America's Test Kitchen}.

\begin{description}

\item[Ingredients:]\ \\
	\begin{itemize}
	\item 1/4 cup salt
	\item 3 pounds bone-in chicken pieces (2 split breasts cut in half crosswise, 2 drumsticks, and 2 thighs), trimmed, or 3 pounds of chicken thighs.
	\item 1	teaspoon vegetable oil
	\item 2	tablespoons unsalted butter
	\item 1	large shallot, minced
	\item 1	garlic clove, minced
	\item 4	teaspoons all-purpose flour
	\item 1	cup chicken broth
	\item 4	teaspoons grated lemon zest plus 1/4 cup juice (2 lemons)
	\item 1	tablespoon fresh parsley leaves
	\item 1	teaspoon fresh oregano leaves
	\end{itemize}

\item[Procedure:]\ \\
	\begin{description}
	\item[Brine the Chicken]\ \\
		\begin{itemize}
			\item	Dissolve salt in 2 quarts cold water in large container. 
			\item Submerge chicken in brine, cover, and refrigerate for 30 minutes to 1 hour. You can brine for longer, if doing so, do the next step only in the last hour of brining.
			\item Mix the zest of one lemon into the brining liquid.
			\item Remove chicken from brine and pat dry with paper towels --- alternatively, put chicken pieces over a rimmed baking sheet fitted with a rack and put in the refrigerator, uncovered, for several hours. Make sure to stretch the skin over the pieces of chicken so that skin covers each piece.
		\end{itemize}
	\item[Brown Chicken on top of stove]\ \\
		\begin{itemize}
			\item Adjust oven rack to lower-middle position and heat oven to 475 degrees. 
			\item Heat oil in oven-safe 12-inch skillet over medium-high heat until just smoking. 
			\item Place chicken skin side down in skillet and cook until skin is well browned and crisp, 8 to 10 minutes. 
			\item Transfer breasts to large plate. 
			\item Flip thighs and legs and continue to cook until browned on second side, 3 to 5 minutes longer. 
			\item Transfer thighs and legs to plate with breasts.
		\end{itemize}
	\item[Preparing the base and Oven-cooking chicken]\ \\
		\begin{itemize}
			\item Pour off and discard fat in skillet. 
			\item Return skillet to medium heat; add butter, shallot, and garlic and cook until fragrant, 30 seconds. 
			\item Sprinkle flour evenly over shallot-garlic mixture and cook, stirring constantly, until flour is lightly browned, about 1 minute. 
			\item Slowly stir in broth, scraping up any browned bits, and bring to simmer. 
			\item Cook until sauce is slightly reduced and thickened, 2 to 3 minutes. 
			\item Return chicken, skin side up (skin should be above surface of liquid), and any accumulated juices to skillet and transfer to oven. 
			\item Cook, uncovered, until breasts register 160 degrees and thighs and legs register 175 degrees, 10 to 12 minutes.		\end{itemize}
	\item[Prepping finishing flavours]\ \\
		\begin{itemize}
			\item	While chicken cooks, chop parsley, oregano, and remaining 2 teaspoon zest together until finely minced and well combined. Squeeze one tablespoon of lemon into a small bowl. 
			\item Remove skillet from oven and let chicken stand for 5 minutes.
		\end{itemize}	
	\item[Finishing Base and Serving]\ \\
		\item Transfer chicken to serving platter. 
		\item Whisk sauce, incorporating any browned bits from sides of pan, until smooth and homogeneous, about 30 seconds. 
		\item Whisk half of herb-zest mixture and the lemon juice into sauce.
		\item Sprinkle remaining half of  herb-zest mixturr over chicken. 
		\item Pour some sauce around chicken. Serve, passing remaining sauce separately.
	\end{description}
\end{description}
\end{document}
