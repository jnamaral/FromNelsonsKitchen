\documentclass[11pt,letterpaper]{article}
\input{headings}
\newcommand \recipeName {Chicken Xinxim}
\chead{\recipeName}

\begin{document}
\input{title}

Chicken Xinxim is a dish traditional of Bahia. It has origen in Africa
and there are many variations of this dish in different regions in
Brazil. I composed this recipe after researching many others. To make
a good xinxim takes time, but most of the work can be done in
advance. Some substitutions are possible, but here I describe the
procedure that I used for the {\em Festa Nordestina} organized by
Brased in November 2008. The best shrimp for this dish is the fresh
frozen shrimp from Thailand. The shrimp must be prepped at least three
days before you will cook the Xinxim, but it can be done even a month
in advance and frozen. I prefer to prepare my own dried shrimp and to
roast and grind my own peanut paste. I find that the dried shrimp
commercially available in ethnic stores have a very punjent smell that
overpowers the dish. The freshly roasted peanuts yield a much more
aromatic peanut paste. However you can substitute good quality peanut
butter for the home-made peanut paste. The red pepper that I use is
{\em Pimenta de Cheiro}, a fragant and not very hot pepper from
Brazil. Make sure to read the whole recipe before you start any
preparation.

\begin{description}

\item[Ingredients:]\ \\
	\begin{itemize}
        \item 2 pounds of skinless chicken thighs 
	\item 1.5 lb of shrimp with shells and heads
	\item 2 large onions
	\item 100 grams of raw peanuts
	\item 2 tablespoons of azeite de dend\^{e} (red palm oil) 
	\item 4 tablespoons of olive oil
        \item 1 cup of all-purpose flour
        \item 1/2 cup low sodium chicken broth
	\item 1 small fresh red hot pepper (mashed)
        \item 4 cloves of garlic (mashed)
        \item 1 inch of ginger (grated)
	\item Juice of 1 lemon
	\item Salt and white pepper
	\end{itemize}

\item[Procedure:]\ \\
	\begin{enumerate}
	\item {\bf Make Dried Shrimp}
	\begin{itemize}
	\item Remove the heads and shell the shrimp, reserve heads and shells for broth (see below).
	\item Devein the shrimp 
        \item Split each shrimp lenghtwise in the same way that you
          would do to butterfly the shrimp, but cut all the way
          through. The best technique is to place the shrimp on a
          clean cutting board, press down with the palm of your hand,
          and slice through with a sharp knive.
        \item rinse the split shrimp throughly in fresh cold water and
          drain on a colander.
        \item dry the shrimp on paper towels.
        \item sprinkle the shrimp with 1 teaspoon of salt
        \item put a rack on top of a cookie sheet and carefully lay
          each shrimp half on top so that they are only touching the
          rack.
        \item place the rack containing the shrimp under a fan in high
          speed and leave it to dry overnight.
        \item turn off the fan and let the shrimp continue to dry on a
          cool dry room for another day or two.
        \item if not ready to use, put in clean ziplock bags and put
          in the freezer.
        \end{itemize}
	\item {\bf Prepare the Shrimp Broth}
	\begin{itemize}
	\item Immediately after shelling the shrimp, add one
          tablespoon of vegetable oil to a hot skillet and saute the
          reserved shrimp shells and heads for a minute or two until
          they change color, add 1 cup of water and let it simmer for
          3 to 5 minutes.
	\item Strain the shrimp-shell broth and reserve. Discard heads
          and shells.
        \item If not ready to proceed, put the broth on a clean
          container and freeze.
	\end{itemize}
	\item {\bf Prepare Peanut Paste}
	\begin{itemize}
        \item Preheat oven to 250F.
	\item Put raw peanut on a cookie sheet and put in the oven.
	\item After 20 minutes start monitoring the peanuts every 5
          minutes or so.
        \item Peanuts may take up to 45 minutes to roast at this temperature,
              but once they get to the point of roasting, they will go from
              perfectly toasted to bitter very quickly.
        \item Peanuts are done when they are very fragant and light golden.
        \item Remove peanuts from oven and let them cool completely.
        \item Process peanuts in the food processor until they form a
          ball of a very smooth paste. You may have to scrape sides of
          the food processor from time to time. It may take 3 to 5
          minutes of processing.
        \item If you are not ready to proceed, put the peanut paste on
          a clean ziplock bag and freeze
	\end{itemize}
	\item {\bf Salt the Chicken}
	\begin{itemize}
	\item Mix 1 tablespoon of salt with 1/4 teaspoon of white pepper.
        \item Make sure to remove any small pieces of bone that may be
          attached to the joints of the chicken thighs.
	\item Wash skinned chicken throughly in running cold water,
          place in a colander and let it drain.
        \item Put chicken in a container with cover to go into the
          refrigerator --- or in a dish that can be covered with
          plastic wrapping.
	\item Sprinkle the misture of salt and white pepper on all
          sides of the chicken thighs.
	\item Cover and put in the refrigerator. It can stay there
          from 2 hours to 2 days.
	\end{itemize}
	\item {\bf Marinate the Chicken}
	\begin{itemize}
        \item One hour before you are ready to start the preparation
          of the dish, mash a clove of garlic and put in the food
          processor along with 1/2 of one onion. Process until it
          turns into a liquid paste. Added this paste to the salted chicken
          thighs.
        \item Add the juice of one lemon to the chicken thighs.
        \item Make sure that all pieces are coated and leave it
          marinating, outside of the refrigerator, for one hour.
         \end{itemize}
        \item {\bf Make Shrimp paste}
	\begin{itemize}
        \item Place the dried shrimp in a bowl.
        \item Pour one cup of hot water over the shrimp.
        \item Cover and let it soak for at least 1/2 hour.
        \item Put re-hidrated shrimp and soaking liquid in the food processor.
        \item Process until you obtain a smooth paste.
         \end{itemize}
        \item {\bf Sautee the Chicken}
	\begin{itemize}
        \item Put flour on a shallow wide dish.
        \item Choose a sautee pan that can accomodate all the chicken later.
        \item Heat up two tablespoons of olive oil in the pan.
        \item Drain the chicken thighs on paper towels.        
        \item Lightly coat chicken pieces with flour.
        \item Sautee chicken until it is golden in each side (you may
          have to work in batches removing golden pieces to a dish).
        \item When all the chicken pieces are golden and out of the
          pan, remove excess fat from the pan, add 1/2 cup of chicken
          broth to deglaze the pan. Let it cook down until it is reduced to 
          a very small amount of liquid.
         \end{itemize}
        \item {\bf Prepare the Sauce}
	\begin{itemize}
        \item Process the remaining 1 1/2 onion in the food processor
          until it is a watery paste.
        \item Add the mashed garlic, grated ginger, and mashed red
          pepper to the reduced juices in the pan and cook for a few
          minutes until fragant.
        \item Add the processed onions to the pan and cook until the
          onions no longer have the strong smell of raw onions.
        \item Add the shrimp broth, the shrimp paste, and the peanut
          paste and stir.
        \item Return the chicken pieces, along with any accumulated
          juices to the pan, nestle chicken into th thick sauce.
        \item Reduce fire to low and simmer gently, watching and
          gently stirring (to not break up the chicken pieces)
          frequently because the thick sauce will stick to the bottom
          of the pan and burn if not watched.
        \item Immediately before serving, drizle the azeite de dende
          on top of the sauce and stir.
        \item If you wish, garnish with a small amount of chopped cilantro.
        \item Serve with steamed white rice.
        \end{itemize}
	\end{enumerate}         
\end{description}
\end{document}



