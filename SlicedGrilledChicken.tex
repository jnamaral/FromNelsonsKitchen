\documentclass[11pt,letterpaper]{article}
\input{headings}
\newcommand \recipeName {Sliced Grilled Chicken}
\newcommand \fileName {SlicedGrilledChicken}
\chead{\recipeName}

\begin{document}
\input{title}

\begin{description}

\item[Ingredients:]\ \\
	\begin{itemize}
	\item Boneless Skinless Chicken Breasts
	\item Lemon
	\item Salt
	\item Sugar
	\item Freshly Ground Pepper
	\item Flavourless cooking oil, such as canola oil
	\item Extra Virgin Olive Oil
	\item Fresh Thyme (or Parsley)
	\end{itemize}

\item[Procedure:]\ \\
	\begin{enumerate}
	\item {\bf Trimming the chicken breasts to make them of a more even for the BBQ (at least three hour before you will BBQ and up to three days ahead.)}
	\begin{itemize}
	\item Remove the tenderloins.
	\item Cut a small triangular shape from the end of each breast to remove the thin end of the breast.
	\item Put each breast on its side and cut a thin slice from the thickest part to make each breast a bit more even.
	\item Wash the chicken breast throughly in cold running water.
	\end{itemize}
	\item {\bf Brine the chicken}
	\begin{itemize}
	\item Make a brining solution with the following proportions: for each  two quarters of water, add 1/4 cup of table salt and 1/4 cup of sugar.
	 \item Add a generous amount of freshly cracked black pepper to the brine.
	 \item Add the zest from one or two lemons to the brine (best to use a lemon zester, but you can also use a small grater). If you zest on top of the water you will capture some of the lemon essential oils into the brine.
	 \item Add all the chicken to the brine and let sit, in the refrigerator, for at least two hours, but you can also keep them in the brine overnight.
	 \item Remove the chicken from the brine and put in a colander or over a rack inside a clean sink to let all the brine run out. 
	 \item If you are doing the brining a day or two ahead, put the drained chicken into the fridge and leave it uncover for several hours to help it keep drying. After 10 or 12 hours, cover the chicken so that it does not get too dry in the fridge.
	 \end{itemize}
	  \item {\bf Finishing the marinating}
	  \begin{itemize}
	 \item About one hour before you plan to start grilling, add some fresh lemon zest, fresh ground pepper, and the juice of one lemon to the chicken.
	 	\end{itemize}
	\item {\bf BBQing the chicken}
	\begin{itemize}
	\item Get your grill as hot as you can (my gas grill gets to 550F)
	 \item Brush the grill with a hard grill brush to make sure it is clean
	 \item Pour a small amount of oil in a small dish, fold a piece of paper towel several times and holding the folded paper towel with kitchen tongues, deep it in the oil and smear all over the grill. Cover the grill to let the oil burn for a minute. Repeat the process three or four times to reduce the stickiness of the grill. 
	\item Grill the chicken, leaving the first side down longer than the second side.
	 \item As each piece of chicken gets done (internal temperature should be between140F and 145F on an instant-read thermometer), remove it to a covered dish (either a bowl covered with a dinner plate or a heavy saucepan with a lead). It is very important to not overcook the chicken as it will continue cooking.
	 \item Keep the cooked chicken covered until it cools enough to handle (20 to 30 minutes). The chicken will release a significant amount of juice while it rests. Make sure to preserve the juice.
	\end{itemize}
	\item {\bf Slicing the chicken to serve}
	\begin{itemize}
        		\item Try to select a cutting board from which it is easy to collect the juices to slice the chicken. As you slice the chicken pour juices back into the container that has the chicken.
		\item Using either a sharp chefs knife or a sharp bread knife, slice the chicken very thinly and put back into the pan with the juices as you slice them.
		\item Add a few tablespoons of extra-virgin olive oil to the chicken.
		\item Chop fresh thyme leaves and  add to the chicken and toss well (best to use hands if making a generous amount of chicken) --- if you do not have fresh thyme, you can use fresh Italian parsley.
		\item Taste to decide if you think it needs more lemon juice or more extra virgin olive oil or more ground pepper.
		\item Serve at room temperature or still slightly warm.
	\end{itemize}
     	\end{enumerate}         
\end{description}
\end{document}



