\documentclass[11pt,letterpaper]{article}
\input{headings}
\newcommand \recipeName {Acaraj\'e}
\chead{\recipeName}

\begin{document}
\input{title}

Acaraj\'e is a dish from Bahia in the Northeast of Brazil. The roots are African. It is a fried dumpling that is made only with black-eyed peas seasoned with onions and fried in dend\^e oil. In Edmonton you find split black-eyed peas at the Excel African Store.

\vspace{0.3in}

\begin{description}

\item[Ingredients:]\ \\
	\begin{itemize}
	\item	500 grams of split black-eyed peas
	\item 1 medium onion 
	\item 1 teaspoon of salt
	\item 3 cups of dend\^e oil  
	\end{itemize}

\item[Procedure:]\ \\

	\begin{enumerate}
	\item {\bf Soak the split black-eyed peas}
	\begin{itemize}
	\item Put the split black-eyed peas in a very deep bowl and cover with plenty of water.
	\item Stir well and let it settle to the bottom of the bowl.
	\item Using a strainer remove any shells that float to the top.
	\item Keep stirring and removing shells until most of them are gone.
	\item Drain the water using a large strainer to separate the peas.
	\item Repeat the washing of the black-eyed peas two more times.
	\item Leave the split black-eyed peas soaking over night.
	\end{itemize}
	
	\item {\bf Prepare the dough}
	\begin{itemize}
	\item Peel and roughly chop the onion.
	\item Put chopped onion in the bowl of a large food processor, or blender, and process until is is a liquid puree.
	\item Drain the split black-eyed peas very well and add to the food processor, or blender, and process until you obtain a smooth paste.
	\item Add the grated onions and salt.
	\item Process to incorporate.
	\item Transfer the paste to the bowl of a standing mixer.
	\item Process in the standing mixer, with the paddle attachment, for several minutes until the mixture double in volume 
	\end{itemize}
	
	\item {\bf Fry the Acaraj\'es}
	\begin{itemize}
	\item Pour the dend\^e oil in a large heavy bottom pan --- you can add non-flavored oil, such as Canola if you do not have enough Dend\^e.
	\item Heat up the oil until it is moderately hot (350 F).
	\item Using two large serving spoons form large clumps of the dough and drop in the hot oil.
	\item Acaraj\'e should fry in moderately hot oil for a fairly long time --- the exact time depend on the size of the acaraj\'e.
	\item Once it got colour in one side, turn the acaraj\'e. Keep turning from time to time until it is a deep mahogany colour.
	\item Serve immediately after frying with vatap\'a and a sauce made of tomatoes, onions, lime, cilantro and peppers.
	\end{itemize}

	\end{enumerate}
\end{description}
\end{document}



