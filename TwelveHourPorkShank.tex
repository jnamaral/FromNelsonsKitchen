\documentclass[11pt,letterpaper]{article}
\input{headings}
\newcommand \recipeName {Twelve-Hour Pork Shank}
\chead{\recipeName}

\begin{document}
\input{title}

Once in a while we buy a whole pig from a local small producer in Edmonton. She raises black Iberico pigs that have a very distinct, sweeter, flavour. When you buy a whole pig, you have to decide what to do with all the parts. I run into this recipe, due to Charla Padilla, from Madison's Grill in Edmonton, published in {\it The Tomato} in 2014. It makes for a delicious dish and most of the cooking is unsupervised. If you would like a simpler, and very delicious dish, check out \href{PorkKnuckle.html}{Pork Knuckle}.


\begin{description}

\item[Ingredients:]\ \\
	\begin{itemize}
	\item 6 pork shanks
	\item	salt and freshly ground pepper
	\item 2 Tbs of unflavoured cooking oil
	\item 1 medium onion, chopped
	\item 2 medium carrots, chopped
	\item 2 celery ribs, chopped
	\item 6 cloves of garlic, minced
	\item 1 cup dry wine
	\item 6 cups vegetable stock (or water)
	\item 3 springs rosemary
	\item 2 bay leaves
	\item 2 springs thyme
	\end{itemize}

\item[Procedure:]\ \\

	\begin{enumerate}
	\item {\bf Browning the Shanks}
	\begin{itemize}
	\item Preheat the oven to 400 F.
	\item Season the shanks with salt and pepper.
	\item Line a rimmed baking sheet with aluminum foil, and fit a rack over the lined sheet.
	\item Place the shanks on the rack and roast in the oven for about 20 minutes or until the shanks are golden brown.
	\item Remove shanks from the oven and reduce the oven to 250 F.
	\end{itemize}

	\item {\bf Make the Flavour Base}
	\begin{itemize}
	\item In a thick-bottomed dish that can accommodate the shanks later, saute the onion, carrots, celery and garlic until softened and lightly browned.
	\item Add the wine and bring to a boil.
	\item Simmer until slightly reduced, about 2 minutes.
	\item Pour in the vegetable stock (or water) and bring it back to a simmer.
	\item Add the rosemary, bay and time.
	\item Place the shanks into the pot so that they are almost submerged. 
	\end{itemize}

	\item {\bf Braise}
	\begin{itemize}
	\item Cover the pan, and braise in a 250 F oven fro 12 hours or until the meat is very tender.
	\item Once it is cooked, transfer the braised shanks to a large deep plater. Cover and keep it warm (you can put it back into the still warm, turned-off oven).
	\end{itemize}

	\item {\bf Finish the Sauce}
	\begin{itemize}
	\item Using a large strainer, strain the braising liquid, pressing hard on the solids.
	\item Return the liquid to the pot and boil until reduced to four cups, about 20 minutes.
	\item Turn off the fire, move the pot to the side, support one side of the pot with something, such as a wooden board, so that it seats tilted. Let stand for about five minutes so that the fat floats to the top.
	\item Spoon off some of the excess fat and discard.
	\item Pour the sauce over the shanks and serve.
	\end{itemize}

	\end{enumerate}
\end{description}
\end{document}



