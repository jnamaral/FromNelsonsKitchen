\documentclass[11pt,letterpaper]{article}
\input{headings}
\newcommand \recipeName {Oven Dried Tomatoes}
\chead{\recipeName}

\begin{document}
\input{title}
 
\begin{description}

\item[Ingredients:]\ \\
	\begin{itemize}
	\item 1 Kg of ripe Roma tomatoes
	\item dried herbs (thyme, oregano, or a  combination)
	\item salt
	\item freshly ground peppers
	\item olive oil
	\end{itemize}

\item[Procedure:]\ \\
	\begin{enumerate}
	\item {\bf Prep de tomatoes}
	\begin{itemize}
	\item Line a baking sheet with parchment paper (parchment paper is best
   but you may also use aluminium foil).
	\item Oil the paper lightly with olive oil.
	\item Slice each ripe tomato lengthwise and run your finger between the
   ridges to remove the seeds and liquid from inside the tomatoes.
	\item Place each half tomato skin side down on the baking sheet. Make
   sure to not overcrowd the pan.
   	\item Sprinkle about 1/2 teaspoon of dried herbs (thyme, oregano, or a
   combination). Crush the herbs in your hands before sprinkling.
	\item Lightly sprinkle the tomatoes with salt and pepper.
	\end{itemize}
	\item {\bf Dry the tomatoes}
	\begin{itemize}
	\item Place in a 220 F over for about 1 to 1 1/2 hour until the edges are
   dry. You do not want to dry the tomatoes too much, they should
   still be a bit moist. You can turn off the oven and leave the
   tomatoes there overnight.
	\item Turn the broil on.
	\item Flip the tomatoes to put their skin side up.
	\item Now you have to pay close attention. Put the tomatoes under the
    broil and watch closely. All you want is for the skins to bubble
    up. You may have to rotate the pan a couple of times.
	\item You can easily remove the skins that have bubbled up.
	\item You may return the tomatoes that have not bubbled up to the oven a
    couple of times in order to remove most of the skins.
    	\item Put the tomatoes in a jar, cover with olive oil and put in
    the refrigerator.
	\end{itemize}
	\end{enumerate}
\end{description}
\end{document}



