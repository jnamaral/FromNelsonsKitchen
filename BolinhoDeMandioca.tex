\documentclass[11pt,letterpaper]{article}
\input{headings}
\newcommand \recipeName {Bolinho Mandioca}
\chead{\recipeName}

\begin{document}
\input{title}

\begin{flushright}
From the book {\it Dona Benta: Comer Bem}.
\end{flushright}


\begin{description}

\item[Ingredients:]\ \\
	\begin{itemize}
	\item Yucca Root
	\item Canola Oil
	\item 2 eggs
	\item 1 teaspoon of baking powder
	\item Salt
	\item Freshly ground black pepper
	\item Water
	\end{itemize}

\item[Ingredients (optional for flavouring, use a selection):]\ \\
	\begin{itemize}
	\item Bacon	
	\item Onion
	\item Parmesan Cheese
	\item Fresh herbs (chives, parsley, basil, rosemary, sage, thyme)
	\end{itemize}

\item[Procedure:]\ \\
	\begin{enumerate}
	\item {\bf Peel the yucca}
	\begin{itemize}
	\item Fill your sink with hot tap water and soak the yucca root in warm water for a few minutes to loosen the skin.
	\item Trim the ends of each yucca root with a chefs knife.
        \item With a pointed paring knife peel the yucca starting at the top. 
	\item Cut the yucca into 3 to 4 inch long segments.
	\item Stand each segment up and cut in half lengthwise.
	\item Cut each half in half again.
	\item Remove the center string from each segment.
	\item Remove any brown, black, stringy or woody parts.
	\item At this point you may submerge the peeled yucca in fresh water so that all the pieces of yucca are covered by water, and put in the freezer, or you may proceed to cook it.
	\end{itemize}
	\item {\bf Cook the yucca}
	\begin{itemize}
	\item If you froze the yucca, drop the frozen yucca with the surrounding water in a pot of fresh water and set it to boil.
	\item Otherwise drop the peeled yucca in the water and bring to a boil. Do not add any seasoning or salt to the water.
	\item Boil for 40-60 minutes until the yucca becomes soft. If it becomes mushy you got a very good yucca. 
	\item Alternatively, specially from frozen yucca, put the yucca in a pressure cooker, bring up to pressure, and cook under pressure for 15 minutes. Put the pressure cooker under running cold water and carefully move the pressure valve to release the steam.
	\item Turn off the fire, add salt to the water and let it stand for 5 to 10 minutes.
	\item Drain the cooked yucca in a colander.
	\item Often the boiled yucca is served as a side dish to meat dishes as is. 
	\end{itemize}
	\item {\bf Mashing the Yucca}
	\begin{itemize}
	\item Let the yucca cool off completely (you may cook it the day before and keep it in the fridge).
	\item Remove the hard string from the center of the pieces of yucca, as well as any hard bits that might be at the end of the cut pieces after they were cooked.
	\item In a large dish, using a fork, mash up the cooked yucca. You want two cups of mashed up yucca.
	\end{itemize}
	\item {\bf Preparing the batter}
	\begin{itemize}
	\item Add two eggs, baking powder, black pepper and flavourings (see below) to the batter and mix very well.
	\end{itemize}
	\item {\bf Frying the Bolinhos}
        \begin{itemize}
	\item Heat up pure canola oil in a deep fryer. Add enough oil for the bolinhos to float in the oil.
	\item Using two Tablespoons, shape the bolinhos and drop in the hot oil.
	\item Cook until they are light golden brown.
	\item You will cook them in several batches
	\end{itemize}
	\item {\bf Flavouring 1: Crisp Bacon}
	\begin{itemize}
	\item Cook the bacon in a sautee pan making sure to not over brown them.
	\item After they are crisp, let it drain on paper towels.
	\item Cut into small pieces and add to the batter.
	\end{itemize}
	\item {\bf Flavouring 2: Fresh herbs}
	\begin{itemize}
	\item Add two tablespoons of grated parmesan cheese and chop a selection of fresh herbs:
		\begin{itemize}
		\item Option 1: chives, parsley, and basil
		\item Option 2: Rosemary and sage
		\item Option 3: two tablespoons of finely chopped onions and thyme.
		\end{itemize}
	\item Serve the bolinhos warm as an appetizer.
	\end{itemize}
	\end{enumerate}
\end{description}
\end{document}




