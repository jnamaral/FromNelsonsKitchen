\documentclass[11pt,letterpaper]{article}
\input{headings}
\newcommand \recipeName {Mint and Bean Pat\^e}
\chead{\recipeName}

\begin{document}
\input{title}

In the Summer of 2017 I decided to research Spanish Tapas to have a Tapas party for our soccer team. I borrowed several books from the University Library and tried several recipes. This one was the best of them all. Its a nice and refreshing taste for a Summer appetizer. The original recipe is with fresh fava beans. I substituted for canned fava beans. I also have done with canned white beans. If using fresh fava beans you will have to boil the beans for about 10 minutes and then, while they are still warm, slip them off their skins. I can't stand the flavour of raw garlic. Thus I replace the one clove of raw garlic for two cloves of garlic dry-roasted in a pan. I also take the edge of the scallions by simply gently warming them in the same pan that I roasted the garlic.
 
\begin{description}

\item[Ingredients:]\ \\
	\begin{itemize}
	\item 1small can of fava beans (12 oz or 350 grams)
	\item 8 oz/225 grams of soft goat cheese
	\item 1 raw garlic clove, crushed or 2 dry-roasted cloves of garlic, smashed.
	\item 2 scallions, finely chopped
	\item rind of a lemon
	\item 2 teaspoons of lemon juice
	\item 1 teaspoon of extra-virgin olive oil
	\item 60 large fresh mint leaves (1/2 oz or 15 grams)
	\item 12 slices of French bread
	\item salt and pepper
	\end{itemize}

\item[Procedure:]\ \\
	\begin{enumerate}
	\item {\bf Preparing garlic and scallions}
	\begin{itemize}
	\item Choose a very small frying pan with a lid
        \item Put the two garlic cloves, with their skins on into the dry pan and cover
	\item Place over moderate heat and monitor every few minutes.
	\item Turn the garlic cloves once they lightly brown on one side.
	\item They are done when the flesh is soft and slightly transparent.
	\item Remove the pan from the hot burner.
	\item Remove the garlic from the pan.
	\item Add the olive oil to the pan.
	\item Immediately add the scallions, stir, and let them warm up in the residual heat for about three minutes. 
	\end{itemize}
	\item {\bf Process the pat\^e}
	\begin{itemize}
	\item Process the beans in a food processor until you obtain a rough paste
	\item Add the cheese, garlic, scallions with oil, lemon rind and lemon juice and process.
	\item Season with salt and freshly grated pepper to taste.
	\end{itemize}
	\item {\bf Serve}
	\begin{itemize}
	\item Put in a bowl
	\item drizzle with extra-virgin olive oil
	\item Serve with the slices of French bread around.
	\end{itemize}
	\end{enumerate}
\end{description}
\end{document}



